\section{Background and Terminology}
To motivate our paper and the space complexity challenge, we first introduce the background and the terminology that we use.  


\header{Matrix Factorization}
Model-based recommender systems approximate the \emph{user-item} matrix \emph{A} through 
the dot-product of the \emph{user}-matrix \emph{U} and the \emph{item}-matrix \emph{V}: $\hat{A} = U * V.$  

The \emph{user-item} matrix \emph{A} is a \emph{nRows}-by-\emph{nCols} matrix.  
Similarly, the approximation matrix $\hat{A}$ also has \emph{nRows} rows, and \emph{nCols} columns.  

The \emph{user} matrix \emph{U} is \emph{nRows}-by-\emph{nDims}: \emph{U} has \emph{nRows} rows, and \emph{nDims} columns.  
\emph{nDims} is the number of latent dimensions.  
The \emph{item} matrix \emph{V} is \emph{nDims}-by-\emph{nCols}.  


\header{Optimizing an Objective Function}
The goal of matrix factorization is to find the best \emph{U} and the best \emph{V} whose dot product optimizes an objective function:
\[ 
\operatorname*{arg\,min (or\,arg\,max)}_{U,V} \left[ f(U, V) = \sum_{i=1}^{nRows} \sum_{j=1}^{nCols} f(\bar{u}_{i}, \bar{v}_{j}) \right]
\]

When we take the gradient of the objective function with respect to a row in \emph{user} matrix \emph{U} (e.g. $\bar{u}_{i}$), 
we sum up the gradient of all the entries in $\hat{A}$ that belong to the same row $\bar{u}_{i}$.  
% equation in slide 10 of cs534L-SAGmf.pptx
$
\frac{\text{d}f(U,V)}{\text{d}\bar{u}_i} = \sum_{j=1}^{nCols} \frac{\text{d}f(\bar{u}_i,\bar{v}_j)}{\text{d}\bar{u}_i}
$

Similarly, when we take the gradient with respect to \emph{V}, we sum up the gradients across different rows that belong to the same column: 
$
\frac{\text{d}f(U,V)}{\text{d}\bar{v}_j} = \sum_{i=1}^{nRows} \frac{\text{d}f(\bar{u}_i,\bar{v}_j)}{\text{d}\bar{v}_j}
$

\header{Gradient Methods in Matrix Factorization.}
