\section{Evaluation}


\subsection{Research Questions}

\header{Quality vs. Time.}
Our first set of research questions focuses on optimization quality vs. time.
Given more time, any gradient method yields a better optimization.  
Our focus here is to identify which gradient method is the most suitable for data-scientists prototyping recommender systems.
In terms of suitability, we mean the gradient method that yields the best quality optimization within the shortest amount of time.
Here, we consider the general \emph{SAG} approach \emph{as is}. 
The next set of research questions studies the specific \emph{space vs. time} trade-off between \tool and the na$\ddot{i}ve$ approach to \emph{SAG}.

Between \emph{SAG}, full deterministic gradient and stochastic gradient,
\begin{sloppy}
\begin{compactenum}
\item Which gradient method yields a better optimization given the same amount of time?
\item Which gradient method uses the shortest amount of time to reach a similar quality of optimization?
\item Can \emph{SAG} and specifically \tool work well with different objective functions in recommender systems?
\item Can \emph{SAG} and specifically \tool work well with different matrix datasets?
\end{compactenum}
\end{sloppy}


\header{Space vs. Time.}
Our second set of research questions investigates whether re-computing is worth the additional time.
Here, we investigate the actual space vs. time trade-off between \tool vs. the na$\ddot{i}$ve approach to \emph{SAG}:

Compared to the na$\ddot{i}ve$ approach to SAG, in practice
\begin{sloppy}
\begin{compactenum}
\setcounter{enumi}{4}
\item How much slower is \tool due to re-computing?
\item How much memory does \tool save?
\end{compactenum}
\end{sloppy}



\subsection{Experimental Setup}

\header{Distinct Objective Functions.}
The objective functions we choose already uses full deterministic gradient (\emph{FG}) or stochastic gradient (\emph{SG}).  
In general, any function that is differentiable, and specifically any function that uses (\emph{FG}) or (\emph{SG}) can use \emph{SAG} and \tool.
If a function is convex, then gradient methods guarantee a global optimum over time.
The functions we have chosen are distinct from each other.  The goal is to illustrate \emph{SAG} and \tool are capable of working with different objective functions.
% cite objective functions
\begin{sloppy}
\begin{compactenum}
\item \emph{Least-squares}: L2 and its variants \cite{mnar, wrmf2008hu, wrmf2008pan} are popular objective functions when building recommender systems.
\item \emph{CLiMF} \cite{climf}: Collaborative-Less-is-More-Filtering uses ordinal logistic regression to smooth the mean reciprocal rank function and to learn how a user ranks different items; 
CLiMF performs gradient ascent because the optimization goal is to maximize an objective function.
\item \emph{BPR-MF} \cite{bpr}: Bayseian Personalized Learning has an objective function that minimizes 
the difference between any two \emph{item} ratings (column entries) of the same user (same row).
BPR-MF performs gradient descent.
\end{compactenum}
\end{sloppy}


\header{Diverse Datasets.}
Our datasets are binary data that serve as implicit feedback in recommender systems. 
They represent diverse relationships including trustees \cite{epinions}, webpage bookmarking \cite{digg12month1}, casting \cite{IMDB}, social network \cite{ljournal2008}, and linking webpages \cite{wikipedia20070206}.
The datasets come from the Sparse Matrix collection at the University of Florida.
% cite datasets
\begin{sloppy}
\begin{compactenum}
\item \emph{Epinions} \cite{epinions}: $A(i,j) = 1$ when user $i$ is a trustee of user $j$, $A(i,j) = 0$ otherwise.  
The trustee relationship is not necesseary mutual.  The epinions dataset is identical to the epinions dataset that Shi et al. used in \cite{climf}. 
\item \emph{Digg12month1} \cite{digg12month1}: $A(i,j) = 1$ when user $i$ tags webpage $j$ as favorable; 0 represents no opinion. 
\item \emph{IMDB} \cite{IMDB}: $A(i,j) = 1 $ if movie $i$ has actor or actress $j$ as cast, $A(i,j) = 0$ otherwise. 
\item \emph{Live Journal} \cite{ljournal2008}:  $A(i,j) = 1 $ if user $i$ has user $j$ as his friend, $A(i,j) = 0$ otherwise. 
The graph is directed because the friendship is not neceseary mutual.
\item \emph{Wikipedia} \cite{wikipedia20070206}: $A(i,j) = 1$ if page $i$ links to page $j$, $A(i,j) = 0$ otherwise.  
\end{compactenum}
\end{sloppy}


\header{Hyper Parameters.}
For the purpose of comparison, we standardize all hyper-parameters across all objective functions, all datasets, and all gradient methods.
The only exception is that we run full deterministic gradient \emph{FG}) for only 500 iterations vs. 5,000 for stochastic gradient (\emph{SG}) and \emph{SAG}.

Convergence theory guarantees that given the same number of iterations, \emph{FG} yields a much better quality optimization than \emph{SG}.
However, our goal is to identify the gradient method that yields the best quality optimization within the shortest amount of time.
Therefore, we want to see whether \emph{FG} would take longer to yield a similar quality of optimization as \emph{SG}, and how much longer.
Through experience with our objective functions and datasets, we observed that 500 iterations \emph{FG} yields a similar quality of optimization as \emph{SG}.
As a result, we run \emph{FG} to 500 iterations, and compare how much longer 500 iterations of \emph{FG} would take than 5000 iterations of \emph{SG}.

\begin{sloppy}
\begin{compactitem}
\item Step size or learning rate: 0.0001
\item Regularization $\lambda$: 0.001; $\lambda$ is identical for regularizing both \emph{user} matrix $U$ and \emph{item} matrix $V$
\item Iterations: 5000 for \emph{SG} and \emph{SAG}, which is roughly 10\% of the number of non-zero entries in each sub-dataset.
\item Latent dimensions ($nDims$): 5 
\end{compactitem}
\end{sloppy}

For gradient descent, step size is $\alpha < 0$ and $\lambda > 0$;
for ascent, step size is $\alpha > 0$ and $\lambda < 0$.


\header{Hardware and OS.}
A MacBook Pro run all experiments that study optimization \emph{Quality vs. Time}.
Our MacBook Pro is the Late 2013 15-inch version \cite{macbookprolo}; it has OS-X Yosemite, 2.3Ghz Intel i7 quad-core CPU, 16GB RAM, and a Nvidia 750M GPU.

When studying \emph{Space vs. Time}, we measure memory usage after the first iteration.  
Initially we plan to run all experiments on the MacBook Pro.
However, the memory-profiing feature of Matlab works only on Windows.  

For the sub-datasets, we run the \emph{memory} experiment on a Dell XPS 12 \cite{dellxps12} laptop.
The Dell XPS 12 has Windows 8.1, 1.6Ghz Intel i5 dual-core CPU, 4GB RAM, and integrated graphics.  

For the full datasets, we run the \emph{memory} experiment on a remote server that has more RAM.
Our remote server has Windows Server 2008R2, 2.50Ghz Intel Xeon 2x quad-core CPUs (total 8 CPU cores) and 16GB RAM.

Both MacBook Pro and remote server have Matlab R2014a; Dell XPS 12 has Matlab R2012a.
All 3 computers have the Matlab parallel computing toolkit.



\subsection{Quality vs. Time}
%\header{Methodology.}
For the purpose of comparison, we fix the seed for generating random numbers so that 
Measure optimization quaity in each iteration. report the best optimixatiom 

