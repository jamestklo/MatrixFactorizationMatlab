\section{Challenge}
As equations \ref{eq:sag_mu} and \ref{eq:sag_mv} illustrate, updating $\bar{m}_{U}^{t+1}$ and $\bar{m}_{V}^{t+1}$ requires $\bar{m}_{entry(b).i}^{t}$ and $\bar{m}_{entry(b).j}^{t}$.  
$\bar{m}_{entry(b).i}^{t}$ and $\bar{m}_{entry(b).j}^{t}$ are the fine-grained gradients of an individual entry $entry(b)$ from the last time (or the most recent time) that $entry(b)$ was sampled.  

When applying \emph{SAG} into matrix factorization, a major challenge is to make these fine-grain gradients available: 
$\bar{m}_{entry(b).i}^{t}$ from equation \ref{eq:sag_mu}, and 
$\bar{m}_{entry(b).j}^{t}$ from equation \ref{eq:sag_mv}  

A na$\ddot{i}$ve approach is to store all these fine-grain graidents.  
As we shall prove, the na$\ddot{i}$ve approach is undesirable because storing all these gradients would take up a lot of space.  

%\newtheorem{totalspace}{}  
\emph{Theorem 1.} 
The total asymptotic space complexity is $\theta(nDims*(min(M,N)+nRows+nCols))$ for storing the fine-grain gradients of all entries that we had previously sampled.  
\begin{proof}
For each individual entry, the amount of space required is $2*nDims$:  
the gradient with respect to row $\bar{u}_i$ ($\bar{m}_{entry(b).i}^{t}$) is a \emph{1}-by-\emph{nDims} row vector;  
the gradient with respect to column $\bar{v}_j$ ($\bar{m}_{entry(b).j}^{t}$) is a \emph{nDims}-by-\emph{1} column vector.  

When we store the fine-grain gradients of all previously-sampled entries, the amount of space required becomes $M*2*nDims$.  
Recalling from the background section, \emph{M} is the number of distinct entries that we previously sampled.  

As shown in equations \ref{eq:sag_mu} and \ref{eq:sag_mv}, \emph{SAG} requires only the most recent gradient of each previously-sampled entry.
Thus for each entry, we store a max of only one set of gradients ($\bar{m}_{entry(b).i}^{t}$ and $\bar{m}_{entry(b).j}^{t}$).
The total amount of space required becomes $min(M,N)*2*nDims$.

Now, according to equations \ref{eq:sag_mu} and \ref{eq:sag_mv}, we must also store the aggregated gradients: $\bar{m}_{U}^{t}$ and $\bar{m}_{V}^{t}$.  
$\bar{m}_{U}^{t}$ takes $nRows*nDims$ space; $\bar{m}_{V}^{t}$ takes $nDims*nCols$ space.  
Thus the total amount of space that we use to store the aggregated gradients is ($nRows*nDims$) + ($nDims*nCols$), which is equivalent to $nDims*(nRows+nCols)$ after simplification. 

Adding the fine-grain gradients and the aggregated gradients together, the asymptotic space complexity becomes $\theta(nDims*(min(M,N)+nRows+nCols))$ after ignoring the constants.  
\end{proof}

\header{No guarantee that lower bound of $M$ is much less than $N$.}
If we can guarantee that $M$ must be much less than $N$ at all times, then the effective asymptotic space complexity becomes $\theta(nDims*(M+nRows+nCols))$; and 
$\theta(nDims*(M+nRows+nCols))$ would be acceptable if $M$ is not much larger than $nRows$ or $nCols$.

Unfortunately, we shall prove that the lower bound of $M$ depends on the convergence rate and error-tolerance, rather than the number of non-zero entries in the dataset.

\emph{Theorem 2.}
The lower bound of $M$, the number of previously-sampled entries, does not depend on $N$, the number of non-zero entries in the dataset.
\begin{proof}

\end{proof}

\header{Chain rule offers no savings in matrix factorization.}
