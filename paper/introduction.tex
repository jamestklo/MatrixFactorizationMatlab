\section{Introduction}
% Why RecSys
Shopping, text advertising, display advertising, renting movies, listening to music… recommender systems are prevalent and ubiquitous in our daily lives.  
% Why Matrix Factorization  
Matrix factorization (\emph{MF}) is a popular technique in model-based recommender systems.  
\emph{MF} has been utilized extensively in past research for handling both explicit \cite{mmmf2005fast, mnar, gapfm} ratings, 
and implicit \cite{wrmf2008hu, wrmf2008pan, climf, bpr, mnar} feedback.  

% Pitfalls with FG and SG  
In recommender systems that utilize matrix factorization, most optimize an objective function.  
Full deterministic gradient (\emph{FG}) and stochastic gradient (\emph{SG}) are the two main gradient methods for optimization.  
All of the recommender systems that we cited above utilize full deterministic gradient, or stochastic gradient.

Unfortunately, both full deterministic gradient and stochastic gradient have pitfalls when it comes to prototyping recommender systems.  
Full deterministic gradient can offer high quality optimizations.  
However, \emph{FG} is slow because in each iteration of optimization, \emph{FG} has to pass through all the samples in the dataset.  
Stochastic gradient is relatively fast; its iteration cost is low because each iteration of \emph{SG} looks at only one or a few samples.  
However, the trade-off with stochastic gradient is that it often provides relatively low quality optimizations.  

High quality optimizations with a low iteration cost is important when building recommender systems.  
The first reason is that data scientists often have to run repeated experiments: e.g. with different objective functions, different metrics, different datasets, and different optimization parameters.  
The second reason is that product life cycles are shortening in the age of agile software engineering.  
Thus data scientists are often faced with the challenge of running more experiments and producing high quality results with less time.  

%Why SAG in prototyping model-based recommender systems  
In this paper, we study the challenge from the perspective of convex-optimization.  
We propose using the stochastic average gradient (\emph{SAG}) method \cite{schmidt2013minimizing, roux2012stochastic} as a viable alternative to using \emph{FG} and \emph{SG} during the prototyping process.  
\emph{SAG} has the distinctive advantage that its optimization quality is proven to be much better than \emph{SG}; at the same time \emph{SAG}'s iteration cost is asymptotically identical to \emph{SG}.  
However, applying and adapting \emph{SAG} to matrix factorization is not trivial because \emph{SAG} requires previously-computed gradients and storing these gradients can lead to very high asymptotic space complexity.  
We explore the challenge with space-complexity, and addresses it by proposing a re-computation approach (\tool) that re-computes the previously-computed gradients.  
\tool preserves the low iteration cost of \emph{SAG}.  
The asymptotic space complexity of our \tool approach is as efficient as the memory-less full deterministic gradient, and stochastic gradient.  

To the best of our knowledge, we are the first to
\begin{compactitem}
\item Identify the pitfalls associated with using full deterministic gradient and stochastic gradient when data-scientists prototype model-based recommender systems.  
\item Propose Stochastic Average Gradient (\emph{SAG}) as a viable alternative for yielding higher quality approximations and low iteration costs in matrix factorization.  
\item Extend \emph{SAG} into \tool for matrix factorization, resolve the space complexity challenge in adapting \emph{SAG} from the domain of large-scale supervised-machine-learning into the domain of prototyping recommender algorithms.  
\item Prove in theory, that our \tool approach has identical asymptotic time complexity and identical asymptotic space complexity as the fast and memoryless stochastic gradient method.  
\item Extensively evaluate and compare \tool across multiple RecSys objective functions and diverse datasets.  
\item Demonstrate in practice that, even without any optimization on the implementation, \tool still yields faster convergence despite the additional time of re-computation, and that \tool uses memory at a level similar to the memory-less stochastic gradient.  
\end {compactitem}
