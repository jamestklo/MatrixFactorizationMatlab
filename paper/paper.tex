\documentclass{sig-alternate}
\pdfpagewidth=8.5in
\pdfpageheight=11in
\usepackage[utf8]{inputenc}
\usepackage[T1]{fontenc}
\usepackage[font=bf]{caption}
\usepackage{listings}
\lstset{language=Python}
\usepackage{color}
\usepackage{xspace}
\usepackage{hyperref}
\usepackage{url,moreverb,graphicx}
\usepackage{ifthen}
\usepackage{paralist}
\usepackage{lipsum}
\usepackage{tikz}
\usetikzlibrary{patterns}
\usepackage{pgfplots}
\usepackage{comment}

\clubpenalty=10000
\widowpenalty = 10000

\definecolor{lightgray}{rgb}{.9,.9,.9}
\definecolor{darkgray}{rgb}{.4,.4,.4}
\definecolor{purple}{rgb}{0.65, 0.12, 0.82}
\lstdefinelanguage{JavaScript}{
  keywords={typeof, new, true, false, catch, function, return, null, catch, switch, var, for, if, in, while, do, else, case, break},
  keywordstyle=\color{blue}\bfseries,
  ndkeywords={class, export, boolean, throw, implements, import, this, assert, forall, exists, not, int},
  ndkeywordstyle=\color{darkgray}\bfseries,
  identifierstyle=\color{black},
  sensitive=false,
  comment=[l]{//},
  morecomment=[s]{/*}{*/},
  morecomment=[l]{\%},
  commentstyle=\color{purple}\ttfamily,
  stringstyle=\color{red}\ttfamily,
  morestring=[b]',
  morestring=[b]"
}

\lstset{
   language=JavaScript,
   backgroundcolor=\color{lightgray},
   extendedchars=true,
   basicstyle=\scriptsize\ttfamily,
   showstringspaces=false,
   showspaces=false,
   numbers=left,
   numberstyle=\scriptsize,
   numbersep=9pt,
   tabsize=2,
   breaklines=true,
   showtabs=false,
   captionpos=b,
   xleftmargin=4.0ex
}

% How about DOMColic for the tool name?
%\newcommand{\tool}{\textsc{DOMColic}\xspace}
\newcommand{\tool}{\textsc{ConcolicDOM}\xspace}

\newcommand{\code}[1]{{\texttt{#1}}}
\newcommand{\js}{{Java\-Script}\xspace}\newcommand{\header}[1]{\par\smallskip\noindent\textbf{#1}}

\newboolean{showcomments}
\setboolean{showcomments}{true}
\ifthenelse{\boolean{showcomments}}
{\newcommand{\nb}[2]{
\fbox{\bfseries\sffamily\scriptsize#1}
{\sf\small$\blacktriangleright$\textit{#2}$\blacktriangleleft$}
}
}
{\newcommand{\nb}[2]{}
}
\newcommand\james[1]{\nb{James}{#1}}
\newcommand\eric[1]{\nb{Eric}{#1}}
\newcommand\ali[1]{\nb{Ali}{#1}}

\renewcommand*{\lstlistingname}{Listing}

\newcounter{head}
\newcommand{\challenge}[1]{%
  \refstepcounter{head}
  \header{\thehead.\ \ #1.}
%\newcommand{\head}[1]{\textbf{#1.}\ \ }
}


\begin{document}

% --- Author Metadata here ---
%\conferenceinfo{ISSTA 2014, Jul 21-26, 2014}{San Jose, California}
%\CopyrightYear{2014} % Allows default copyright year (20XX) to be over-ridden - IF NEED BE.
%\crdata{0-12345-67-8/90/01}  % Allows default copyright data (0-89791-88-6/97/05) to be over-ridden - IF NEED BE.
% --- End of Author Metadata ---


\title{{\ttlit ConcolicDOM:} Concolic Generation of DOM Structures for Unit Testing JavaScript}
\numberofauthors{3} 
\author{
\alignauthor
James Lo\\	
       \affaddr{Computer Science}\\
       \affaddr{University of British Columbia}\\	   
       \affaddr{Vancouver, Canada}\\
       \email{tklo@cs.ubc.ca}
\alignauthor
Eric Wohlstadter\\
       \affaddr{Computer Science}\\
       \affaddr{University of British Columbia}\\	   
       \affaddr{Vancouver, Canada}\\
       \email{wohlstad@cs.ubc.ca}
\alignauthor
Ali Mesbah\\
       \affaddr{Elec. and Comp. Engineering}\\
       \affaddr{University of British Columbia}\\	   
       \affaddr{Vancouver, Canada}\\
       \email{amesbah@ece.ubc.ca}
}
%\date{24 January 2014}

\maketitle
\begin{abstract}
%As Web applications become more prevalent in our daily lives, quality assurance of Web applications has also become more important.  
Considerable JavaScript code is written to access and update a Web application's user interface through the Document Object Model API.  
The DOM models the UI in a tree structure.  
In this paper we present our generic and browser independent approach for concolic generation of DOM trees for testing JavaScript Web applications.
Testing Web applications remains a challenge because executing different parts of JavaScript code requires different yet specifically precise DOM tree structures.
If there is any mismatch between the code and the DOM, e.g. when a DOM operation fails, entire code execution would eventually halt and the test would terminate prematurely.  
To overcome these challenges, we apply concolic techniques to generate HTML.  
We designed a DOM solver to support the 2D structure of the DOM tree, to infer implicit clues from DOM operations that are partial and incomplete, and to tailor the DOM tree for targeting precise subsets of the code base.  
We also implemented an end to end automatic system from deducing constraints to generating HTML and driving test execution because the number of unique DOM trees can grow as exponentially as the number of execution paths.
We conducted a case study on the DOMtris application in which we will show how our approach significantly improves path coverage that includes a part of JavaScript code that implements a core functionality of the application.
\end{abstract}

% A category with only the three required fields
\category{D.2.5}{Software Engineering}{Testing and Debugging}%[Symbolic execution, Test coverage of code, Test execution]
%A category including the fourth, optional field follows...
%\category{D.3.2}{Software}{Programming Languages}[JavaScript]

\terms{Design, Algorithms, Experimentation}

\keywords{Testing, concolic execution, DOM, JavaScript}

\section{Introduction}
%JavaScript is increasingly a popular language for software implementation. 
%; and the World Wide Web is increasingly an attractive platform for delivering applications.
%For end users, HTML5 and its standardization enable Web apps to have an interactivity and feature-richness comparable to those implemented for traditional desktops.  
%The latest round of browser wars makes executing JavaScript more efficient, robust, secure and consistent.  
%For programmers, JavaScript is relatively easy to adopt, and JavaScript does not have the burden of memory management and static typing.
%More operating systems in both the desktop~\cite{chromeApps, windows8javascript} and mobile~\cite{apacheCordova, iosWebView, blackberryWebWorks, firefoxOS, androidWebView, tizen} actually now support installing and running JavaScript apps on the OS similar to native apps.
%CSS3 and the responsive UI paradigm enable UI designers to use a single HTML and still support multiple devices of different screen sizes.  
%The Bring Your Own Device (BYOD) movement in Enterprise IT increases hardware heterogeneity, which also makes JavaScript apps\footnote{JavaScript apps are preferred in Web browsers because they are lighter weight than Java applets and they do not require installation of any proprietary plugins such as Flash and Silverlight} 
%a conveniently portable solution for delivering the application front end (e.g.~\cite{BNSFoffice365}).
%Emergence and scalability of Node.js also make JavaScript widely adopted on the server side.  
%Consequently, many institutions such as the Khan Academy~\cite{khanAcademy} use JavaScript for teaching programming; and JavaScript has consistently been a top 2 in the RedMonk~\cite{redmonk} popularity rankings.%~\footnote{results are based on projects hosted at GitHub and questions asked at StackOverflow}.
%
JavaScript has become a popular language for application development. 
To create responsive web applications, developers write \js code that interacts dynamically with the Document Object Model (DOM). 

The DOM is a tree-like structure that provides APIs \cite{w3dom} for accessing, traversing, and mutating the content and structure of HTML elements at runtime. As such, changes made through \js code via these DOM API calls become directly visible in the browser. 

The complex interplay between JavaScript and the DOM makes it particularly challenging for developers to understand \cite{clematisICSE14}, analyze~\cite{staticJsFSE11, staticJsFSE13}, and test~\cite{artemis, pythia} \js applications. Recent empirical studies have also shown that these JavaScript-DOM interactions are particularly error-prone~\cite{frolin2013}.

In order to unit test a \js function that has DOM operations, a DOM instance needs to be provided in the exact structure as expected by the function. Otherwise, the execution of the function fails (e.g., due to null exceptions) and the test case terminates prematurely.
To avoid this problem, testers typically write test fixtures for the DOM structure required in their \js unit tests. This manual process is, however, time-consuming and costly.

Current automated web testing techniques ignore the DOM as a test input and mainly focus on generating (1) event-sequences and UI inputs, e.g., strings for form input fields \cite{mesbah:tse12, kudzu}, or (2) test inputs in the form of function arguments~\cite{artemis, pythia, jalangi}.

In this paper, we provide a fully automated technique for generating the DOM structure needed for unit testing \js functions. Our technique works through a combination of (1) dynamic backward slicing of \js code, (2) inferring DOM constrains along different paths of the function, and (3) solving those DOM constraints through a DOM-aware constraint solver. We present our concolic engine called \tool, which iteratively analyzes \js code and generates satisfiable DOM structures to achieve runnable test cases with high code coverage. 

 To the best of our knowledge, our work is the first to consider the DOM as an important test input artifact and generate satisfied DOM structures for \js unit testing.


%Random testing (e.g.~\cite{artemis}), mutation testing (e.g.~\cite{pythia}), concolic testing (e.g.~\cite{eventConcolic, feedbackConcolic, kudzu, jalangi, cute})... to our best knowledge, almost all of existing research on test generation focused on generating 1) function arguments for unit testing, or 2) events and UI inputs (e.g strings for text fields and forms; mouse clicks for buttons and selection boxes; and key presses) for application-level testing.  
%However, having just the function arguments, events and UI inputs is often insufficient.  
%For example, in a Web app, a properly satisfied dependency such as the DOM is often necessary for test cases and assertions to reach complete execution.  
%
% To seamlessly update the client-side state of the application, developers write \js code that calls these DOM APIs. 
%\tool augments approaches that aim to generate tests automatically.  
%JavaScript interacts with the DOM and DOM-related \javascript faults are prevalent 
%For example, because HTML describes the graphical user interface of a Web app, considerable JavaScript code is written to access and mutate HTML through the Document Object Model (DOM) API.  
%When JavaScript code runs, its runtime execution would encounter operations on the DOM API (DOM operations) that would subtly imply the DOM tree (and thus the webpage's HTML) to have a particular structure.  
%In other words, when trying to run a test case, if the DOM structure does not satisfy what the code expects it to be, execution would fail and the test case would terminate prematurely.  
%Indeed, a recent empirical study~\cite{frolin2013} of bug reports has concluded that a majority of bugs in JavaScript Web applications are DOM related.  
%Being able to fully test JavaScript code that contain DOM operations would be critical in assuring the quality of a Web application.  
%\header{Contributions.}
%The following are the main contributions of the Research Proficiency Evaluation (RPE) report:

Our work makes the following key contributions:

\begin{compactitem}
\item We propose a generic, automated concolic approach for generating DOM tree structures to effectively unit test JavaScript code. %that contains DOM operations.
\item We describe a technique for deducing DOM structure constraints based on dynamic backward slicing of \js code.
\item We present a novel method for solving the deduced DOM constraints. The method supports the DOM's 2D tree structure, infers implicit clues from DOM operations that are partial and incomplete, and tailors the DOM tree for targeting precise subsets of the code base.
\item We present an implementation of our approach called \tool, which is browser-independent and fr\-eely available. \tool has an end-to-end concolic engine that supports deducing and solving DOM constraints, generating HTML test fixtures, and driving test execution.
\item We report an empirical evaluation of our approach in which we assess the efficacy of \tool in improving test execution completion and coverage.  %If a function cannot be fully executed, the test case's assertions cannot be fully run.  
\end {compactitem}




% present emprical results

% cannot generate HTML for game, 
%Note that the code does not imply the following, which we take for granted in a Tetris game.
%\begin {compactitem}
%\item Each {\tt row} is a child element of {\tt field}.
%\item Each {\tt row} is vertically stacked: {\tt row10} is right above {\tt row9}, which is also right on top of {\tt row8}, and so on.
%\item Children of a {\tt row} are blocks that make up a piece.  
%\item etc.
%\end {compactitem}

%% intro
\begin{comment}

1. JavaScript interacts with the DOM and DOM-related \javascript faults are prevalent 
2. Unit testing JavaScript functions required the DOM as input, not any DOM but the DOM in a specific structure
3. Other techniques only focus on input values to functions and ignore the DOM
4. Our technique addresses this important issue by...
5. Our contributions are... our results show that...

\end{comment}

\section{Challenge}
As equations \ref{eq:sag_mu} and \ref{eq:sag_mv} illustrate, updating $\bar{m}_{U}^{t+1}$ and $\bar{m}_{V}^{t+1}$ requires $\bar{m}_{entry(b).i}^{t}$ and $\bar{m}_{entry(b).j}^{t}$.  
$\bar{m}_{entry(b).i}^{t}$ and $\bar{m}_{entry(b).j}^{t}$ are the fine-grained gradients of an individual entry $entry(b)$ from the last time (or the most recent time) that $entry(b)$ was sampled.  

When applying \emph{SAG} into matrix factorization, a major challenge is to make these fine-grain gradients available: 
$\bar{m}_{entry(b).i}^{t}$ from equation \ref{eq:sag_mu}, and 
$\bar{m}_{entry(b).j}^{t}$ from equation \ref{eq:sag_mv}  

A na$\ddot{i}$ve approach is to store all these fine-grain graidents.  
As we shall prove, the na$\ddot{i}$ve approach is undesirable because storing all these gradients would take up a lot of space.  

%\newtheorem{totalspace}{}  
\emph{Theorem 1.} 
The total asymptotic space complexity is $\theta(nDims*(min(M,N)+nRows+nCols))$ for storing the fine-grain gradients of all entries that we had previously sampled.  
\begin{proof}
For each individual entry, the amount of space required is $2*nDims$:  
the gradient with respect to row $\bar{u}_i$ ($\bar{m}_{entry(b).i}^{t}$) is a \emph{1}-by-\emph{nDims} row vector;  
the gradient with respect to column $\bar{v}_j$ ($\bar{m}_{entry(b).j}^{t}$) is a \emph{nDims}-by-\emph{1} column vector.  

When we store the fine-grain gradients of all previously-sampled entries, the amount of space required becomes $M*2*nDims$.  
Recalling from the background section, \emph{M} is the number of distinct entries that we previously sampled.  

As shown in equations \ref{eq:sag_mu} and \ref{eq:sag_mv}, \emph{SAG} requires only the most recent gradient of each previously-sampled entry.
Thus for each entry, we store a max of only one set of gradients ($\bar{m}_{entry(b).i}^{t}$ and $\bar{m}_{entry(b).j}^{t}$).
The total amount of space required becomes $min(M,N)*2*nDims$.

Now, according to equations \ref{eq:sag_mu} and \ref{eq:sag_mv}, we must also store the aggregated gradients: $\bar{m}_{U}^{t}$ and $\bar{m}_{V}^{t}$.  
$\bar{m}_{U}^{t}$ takes $nRows*nDims$ space; $\bar{m}_{V}^{t}$ takes $nDims*nCols$ space.  
Thus the total amount of space that we use to store the aggregated gradients is ($nRows*nDims$) + ($nDims*nCols$), which is equivalent to $nDims*(nRows+nCols)$ after simplification. 

Adding the fine-grain gradients and the aggregated gradients together, the asymptotic space complexity becomes $\theta(nDims*(min(M,N)+nRows+nCols))$ after ignoring the constants.  
\end{proof}

\header{No guarantee that $min(M,N)$ is small.}
If we can guarantee that $min(M,N)$ is small, or that $min(M,N)$ is asymptotically not larger than $nRows$ or $nCols$, 
then the effective asymptotic space-complexity becomes $\theta(nDims*(nRows+nCols))$, which is the most compact anyone can possibly get.  Unfortunatley, we shall prove that there is no such guarantee.  

First, we explore what the best possible asymptotical space-complexity can be in matrix factorization.

\emph{Theorem 2.}
$\Omega(N+nDims*(nRows+nCols))$ is the lower-bound asymptotic space-complexity in matrix factorization.
\begin{proof}  
Matrix factorization is to approximate a matrix $A$ (e.g. the \emph{user-item} matrix) through the dot product of two matrices $U$ (e.g. the \emph{user} matrix) and $V$ (e.g. the \emph{item} matrix). 
$A$ has $N$ non-zero entries.  $U$ is a $nRows$-by$nDims$ matrix; $V$ is a $nDims$-by-$nCols$ matrix.  
In each iteration of convex optimization, we must update $U$ and $V$, and use an objective function to compare our approximation to the ground-truth matrix $A$.  
Therefore, any matrix factorization algorithm would have an asymptotic space-complexity of at least $\Omega(N+nDims*(nRows+nCols))$.
\end{proof}

If we can guarantee that $min(M,N)$ is asymptotically not larger than $nRows$ or $nCols$, 
then we can prove that the na$\ddot{i}$ve approach has already achieved the best possible asymptotic space-complexity, and that our challenge is irrelevant.
However, we shall prove that such guarantee does not exist.

\emph{Theorem 3.}  There is no guarantee that $min(M,N)$ is asymptotically not larger than $nRows$ or $nCols$.
\begin{proof}
$N$ is the number of non-zero entries in the matrix $A$.
Unless there is, or unless we are restricted to an upper-bound of sparsity, then $N$ must have $O(nRows*nCols)$ space.

$M$ is the number of \emph{distinct} entries that we previously sampled.
According to equation \ref{eq:msampled}, $M$ depends on the batch size at each iteration $B_r$, and the number of iterations previously done $t-1$.
Usually, the batch size is a constant $B$.  Thus the lower bound of $M$ most likely depends on the lower bound of $t$.
However, the lower bound of $t$ depends on the convergence rate, and the tolerance of error $\epsilon$.
For example, if the convergence rate is exponential (e.g. $O(p^t)$), then the loewr bound of $t$ is $\Omega(log(\frac{1}{\epsilon}))$.
%$p$ is a constant calculated from the Hessians of the objective function; $0 \leq p \leq 1$.
%Convergence rate depends on the combination of both the underlying objective function, and the gradient method being used.
%For example, for a given objective function, both full deterministic gradient and stochastic average gradient may yield the fast exponential convergence $O(p^t)$.
Therefore, the lower bound of $M$ does not depend on $N$, $nRows$ or $nCols$.
Given a dataset, the only way to enforce $M \leq N$ is to either tolerate a high error, or to find a combination of objective function and gradient method that yields the fastest convergence rate as possible.
The asymptotic space-complexity of using \tool does not depend on $M$.
Thus \tool does not enforce data-scientists to tolerate a high error.
Given any objective function, the convergence rate of \emph{SAG} \cite{schmidt2013minimizing, roux2012stochastic} is always faster than stochastic gradient and is sometimes as fast as the fastest full deterministic gradient.
\tool preserves the convergence rate of \emph{SAG}.
\end{proof}

\header{Chain rule offers no savings in matrix factorization.}
In supervised machine-learning, we can use the chain-rule in differential-calculus to reduce space-complexity.
Unfortunately, applying the chain-rule in matrix-factorization would result in a space-complexity larger than the na$\ddot{i}$ve approach.


\begin{equation} \label{eq:sml}
\operatorname*{arg\,min (or\,arg\,max)}_{\omega} f(X, \omega)
\end{equation}
\section{Approach}
%Given a piece of JavaScript code, and a path that we intend the code execution to follow,  our goal is to generate a DOM tree to guide and support the code execution.  

At a high-level, our approach concolically generates DOM trees to achieve path coverage. We first instrument the \js code to trace the execution. Next, we identify dynamic backward slices of DOM operations and deduce DOM-specific constraints from the slices. Then, We feed these constraints into a DOM-aware solver for generating the expected DOM structure. 
The resulting DOM tree is then transformed into HTML and used as a test fixture in a \js unit test. We continue this process recursively until all DOM-based paths in the \js function are covered.

We present an overview of some challenges in generating satisfiable DOM trees, followed by a salient description of our approach.

\subsection{Challenges}

\header{Multiple Constraints.}
Each DOM operation provides a single constraint or clue to a subset of the overall DOM tree structure. A simple approach would be to generate DOM elements {\tt "}just in time{\tt "}.  
However, such na\"ive approach does not work when there are multiple DOM constraints in the logic of the code. 
For example when the index page of Wikipedia~\cite{wikipedia} is loaded,  a jQuery function retrieves the DOM element with a specific ID: {\tt \$("\#B13\_120517\_dwrNode\_enYY")}.
To satisfy this DOM operation, we can simply create a DOM element with ID {\tt "B13\_120517\_dwrNode\_enYY"}.  
However, as the webpage continues to load, there is call made to {\tt \$("div\#B13\_120517\_dwrNode\_enYY")}.  
This time the DOM is expected to have a \code{<div>} element by the exact same ID.  
The second DOM operation can be satisfied {\em only} if we created a \code{<div>} element in the first place.  Thus, DOM operations have to be collectively analyzed for proper constraint inference.

\begin{figure}[t]
\begin{lstlisting}[caption=Example code showing how DOM operations can have logical constraints that are interdependent with each other: {\tt line 3} and {\tt line 6}.  To make all these {\tt if} statements {\tt true} the sub conditions in {\tt line 3} and {\tt line 6} become mutually exclusive: they cannot be {\tt true} at the same time because {\tt d} cannot be both a parent and a child of the same DOM element {\tt elem}.   Note that the final 2 conditions ({\tt line 9} and {\tt line 12}) would collectively influence the DOM solver to decide which sub condition ({\tt line 3} vs. {\tt line 6}) to become {\tt true}.,label=domOr]  
if (d === elem.firstElementChild
 || d === b.lastElementChild) {}
... 
if (d === elem.parentElement
 || d === b.parentElement) {}
...
if (b.previousElementSibling === 
    c.firstElementChild) {}
... 
if (elem.parentElement.parentElement 
    === c.lastElementChild.previousElementSibling) {}  
\end{lstlisting}
\end{figure}


\header{2D Tree Structure.}  
The DOM tree has a 2-dimensional hierarchical structure that makes it a more challenging target to reason about than traditional single dimensional data types (e.g., integers, real numbers, or strings). 
For instance, when given the constraint \code{(a.previousElementSibling === c.firstElementChild)}, we must infer that \code{c} is the parent of \code{b}. 
\sloppy{
This type of aliasing can become quite complex because DOM operations can be chained and multiple sub-chains of varying length can refer to the same DOM element.
%In JavaScript, a DOM operation on a DOM element (e.g. {\tt elem.parentElement}) returns another DOM element.
%Thus chaining occurs when more DOM operations build on an existing DOM chain.
For example, when we extend the chain {\tt elem.parentElement} with another {\tt .parentElement} DOM operation, {\tt elem.parentElement.parentElement} returns the grandparent of {\tt elem}.
To solve the constraint {(\tt elem.parentElement.parentElement === c.lastElementChild.previousElementSibling}) (In \autoref{domOr}, {\tt line 12}), one has to infer that {\tt b.nextElementSibling} is also {\tt c.lastElementChild} (see \autoref{trees} Solved DOM), because the code also stated that ({\tt d === b.lastElementChild}) and ({\tt d === elem.parentElement}).  
Note that such chains can span in both parental and sibling dimensions. %, e.g., {\tt elem.parentElement.parentElement.nextElementSibling} {\tt .children[2].previousElementSibling}.    
}



 






\begin{figure}
\begin{lstlisting}[caption=Example showing how code is instrumented for dynamic analysis.  The comment at line 9 shows the decorated object {\tt a} and its nested tree data structure.    
{\tt a}'s actual value is {\tt true} because both left and right hand side have the same value 10: {\tt line 11} and {\tt line 12},label=sheq]  
// Before Instrumentation
var row = getElementById("row"+i);
var a = row.children.length === b; 
if (a) {}

// After Instrumentation(i)
var row = _CALL(getElementById, _ADD(_CONST("STRING filename.js 0", "row"), i));
var a = _SHEQ(_GET(_GET(row, "children"), "length"), b);
/* a = {val: true
      , op:	_SHEQ
      , 0:	{val: 10, op:_GET, ...}
      , 1:	{val: 10, ...}}; */
if (__cond("IF_NAME filename.js 1", a)){}
\end{lstlisting}
\end{figure}

\subsection{challenges}

\section{Inferring DOM Constraints}

\header{Decorated Execution.}
Decorated execution is where we instrument the JavaScript code so that the execution of each JavaScript operator can be captured and decorated with additional data for producing a dynamic trace and a dynamic backward slice.  
Sample code \ref{sheq} illustrates the semantics of decorated execution.  
A general rule of thumb is that we transform each use of a JavaScript operator (e.g. {\tt .}) into a call to a corresponding operator function (e.g. {\tt \_GET()}).  
For example, {\tt row.children} becomes {\tt \_GET(row, "children")}.  {\tt \_SHEQ} represents the strict equal operator ({\tt ===}).  
Each operator function wraps (thus decorates) the actual computed value inside a decorated object that also contains other data for tracing and slicing.   

A special case happens when we transform the {\tt \&\&} and {\tt |}{\tt |} ({\tt or}) operators, in which we have to consider the precedence of the operator's left hand side.   
For example, if the code is ({\tt a \&\& a.b}), the transformed version becomes {\tt \_AND(a, a.b)}; yet we do not want to execute {\tt a.b} when {\tt a} is {\tt null} or {\tt undefined}.  
A possible solution is to reuse {\tt a}: {\tt \_AND(a, a \&\& \_GET(a, "b")}.  
However, the left hand side can be a call to a function that may change the internal state of the application: e.g. {\tt appendLog() \&\& update()}.
Thus our solution is to assign the left hand side into a temporary variable, and then check the value of the temporary variable before executing the right hand side: 
{\tt \_AND(t = a, t \&\& \_GET(a, "b"))} and {\tt \_AND(t = \_CALL(appendLog), t \&\& \_CALL(update))}.  

Another special case is the {\tt ++} and {\tt ---} operators.  
For example, with {\tt i++} we have to first assign the original value of {\tt i} to a temporary variable before incrementing {\tt i}, then we return the temporary variable.
% functions, boundary to native functions


\header{Backward Slicer \& Post Order Traversal.}
% would a diagram be good?
The data structure of the decorated objects can be seen as a nested or tree structure because the calls to the operator functions are nested inside one another.  
For example, in Sample Code \ref{sheq}, the call to {\tt \_GET(..., "length")} is nested inside the call to {\tt \_SHEQ()}.  
Therefore, if we simply put the name of the operator function (e.g. {\tt "\_GET"}, {\tt "\_SHEQ"}, ...), inside the trace data, we can easily generate a backward slice via a tree traversal.  

In Sample Code \ref{sheq}, the variable {\tt a} equals to ({\tt row.children.length === b}).  
Thus {\tt a}'s backward slice must contain the backward slice of {\tt b} and the backward slice of {\tt row.children.length}, linked by the strict equal ({\tt ===}) operator.  
At line 8, the decorated object returned by {\tt \_SHEQ()}, assigned to the variable {\tt a}, is the tree parent of 2 decorated objects: {\tt b}, and the decorated object returned by the {\tt \_GET()} call.  

The tree children of a decorated object always come from earlier executions, e.g. {\tt \_GET(..., "children")} is executed before {\tt \_GET(..., "length")} before {\tt \_SHEQ(..., b)}.
Thus the tree's hierarchical structure is reversely proportional to the temporal order in which the operator functions are executed.  

During the traversal, we identify conditions that contain DOM operations and extract these DOM operations accordingly.  
In a chain of DOM operations, the operations closer to the chain head always come from earlier executions, thus the tree's hierarchy is also reversely proportional to the chaining order of DOM operations.  
The backward slicer traverses the decorated objects in post order, which is bottom up from leaf to root.  
This way, the dynamic backward slice not only yields a temporal history of the code's runtime execution, it also conveniently traces the DOM operation chains in the order from head to tail.

Each tree leaf represents an input or a constant.  
For example, a dynamic backward slice of {\tt row} would lead us to the DOM element with ID {\tt "row"+i}, where {\tt "row"} is a constant string, 
and {\tt i} has a backward slice leading to {\tt field.children.length}, which would lead us to the DOM element with ID {\tt "field"}.  
Because variables can be used multiple times, each variable can belong to more than 1 tree and can have more than 1 parent.  
Thus the data structure would appear more like a directed acyclic graph than a tree, even though a variable would never be a tree ancestor of any of its own ancestors.  

\begin{figure}
\begin{lstlisting}[caption=Constraints for generating a DOM tree that would satisfy for going the {\tt True} branch in the {\tt if} statement of Sample Code ~\ref{dom0}.  The constraints are shown in the input format for the CVC~\cite{cvc3} implementation of the SMT solver. {\tt \%} is the comment operator in CVC.,label=constraints0]
% document.getElementById("field");
% document.getElementById("row"+0);
ASSERT DISTINCT(field, row0);

% (field.children.length)--;
ASSERT childrenLength(field) > 0;

% row.children.length === 10;
ASSERT childrenLength(row0) = 10;
\end{lstlisting}
\end{figure}


\header {Trace Mapper \& Constraints Deducer.} 
For each instance that a condition is executed, the backward slicer would yield what DOM operations the instance has and how these DOM operations are related or interdependent on one another.  
Because each condition can get executed more than once at different time points, the MapDeducer would aggregate all executed conditions, map them according to their ID, and deduce constraints for the DOM solver to generate a satisfiable DOM tree.  
So far everything is code-oriented in which we focus on each condition and its dynamic backward slice.  The MapDeducer would transition the focus to be DOM-oriented in which we assemble clues about the same part of the DOM tree that are scattered across multiple lines of code.  
The MapDeducer would put together the processed clues across multiple parts of the DOM tree back together, into a single set of constraints for the DOM solver to generate a satisfiable DOM tree.

Sample Code~\ref{constraints0} illustrates constraints for going to the {\tt true} branch of the {\tt if} statement in Sample Code~\ref{dom0}, resulting in Sample Code ~\ref{html0}.  
If we want to go to the {\tt false} branch, e.g. {\tt ASSERT NOT(childrenLength(row0) = 10)}, then the solver would generate a number of children not equal to 10 for {\tt row0}.  The exact number of children has not been deterministic based on our experiments: sometimes {\tt row0} has 2 children, sometimes {\tt row0} has none.  
%Recall the jQuery example from the Challenges section.  The MapDeducer would 

\begin{figure}
\begin{lstlisting}[caption=Example HTML generated from the results of the DOM solver based on the constraints defined in Sample Code ~\ref{constraints0}.  Note that {\tt row0} is not a child of {\tt field} because the source code in Sample Code \ref{dom0} did not require the rows to be children of {\tt field}.,label=html0]  
<span id="field"><span></span></span>
<span id="row0">
  <span></span><span></span>
  <span></span><span></span>
  <span></span><span></span>
  <span></span><span></span>
  <span></span><span></span>
</span>
\end{lstlisting}
\end{figure}

\section{DOM Solver}
The DOM solver takes the constraints defined by the MapDeducer and attempts to generate a satisfiable DOM structure.  The solver is implemented as an extension of a SMT solver~\cite{cvc3} and would report anything not satisfiable.  

\header{DOM Tree \& DOM Operations.}
Recall a major part of the DOM is its single parent, multi-children tree structure.  When generating a satisfiable DOM, we use the execution of DOM operations to infer the overall DOM tree.    
Each DOM operation in any line of code is like a piece of a puzzle describing a subset clue of the overall DOM tree.   
For example {\tt a = elem.parentElement.nextElementSibling} implies 2 subset clues: {\tt elem} has a parent element, and the parent has a sibling.  
Note that when the condition is {\tt a = elem.parentElement.nextElementSibling === null}, then the clues become {\tt elem} has a parent element, yet the parent has no next sibling and thus is the last child.

That said, questions remain unanswered about exactly where does {\tt elem} fit in or belong in the overall DOM tree; and other DOM operations would provide clues for that.  
The DOM solver would take all the clues and generate a satisfiable tree structure.   

\begin{figure}
\begin{lstlisting}[caption=HTML generated for guiding the execution to follow the {\tt true} branch in the {\tt if} statement in Sample Code\ref{dom0}.label=htmlExtended]  
<span id="c">
  <span/>
  <span id="b">
    <span id="d">
      <span id="elem"/>
    </span>
  </span>
  <span/>
</span>
\end{lstlisting}
\end{figure}


% XPath
% how to create rules in DOM solver, example rules.  Intuition behind it.
% why not XML solver, not scalable
% SMT-lib language, swappable between CVC and Z3.
	% take advantage of Moore's law in terms of hardware performance and breakthroughs in constraint solvers
	% future compatibility for multiple data types

\header{DOM Operations into SMT Quantifiers.}	
In the solver we transform each DOM operation into a SMT function.  We then use quantifiers (e.g. {\tt EXISTS}, {\tt FORALL}) to define how the SMT functions relate to each other.  
Sample Code ~\ref{childrenLength} shows the boolean functions and integer functions we defined for supporting the {\tt elem.children.length} operation.  We first quantify the parent-child relationship: 
\begin{compactitem}
\item a node cannot be a child of itself, see {\tt line 1} and {\tt line 4} in Sample Code ~\ref{childrenLength}.
\item a child of a node cannot be the node's parent at the same time: {\tt line 8}.
\item a child can have only 1 parent: {\tt line 13}.
\end{compactitem}
Next, we define how children are ordered and quantify {\tt children.length}:
\begin{compactitem} 
\item first child starts at position or index 0: {\tt line 18} and {\tt line 24}.
\item last child has the largest child index: {\tt line 30}.
\item {\tt children.length} equals to one plus the child index of the last child, because the first child starts at position 0: {\tt line 40}.
\end{compactitem}
The parent SMT functions are quantified as the inverse of the child SMT functions (e.g. {\tt line 48}).  Similar to {\tt firstChild()} and {\tt lastChild()}, the sibling SMT functions are defined by extending {\tt children(x, y, j)}.  
For example, the next sibling of a node has the same parent and child index {\tt j+1}, when the node has child index {\tt j}.

\begin{figure}[th]
\begin{lstlisting}[caption=SMT functions for defining the children.length DOM operation.  We start with defining the parent-child relationships; then move on to the ordering of children; then use the child index of the last child to define and quantify the {\tt childrenLength()} boolean function,label=childrenLength.]  
% child(x, y): x is a child of y.
child: (Node, Node) -> BOOLEAN;	

% x cannot be a child of itself.
ASSERT FORALL (x: Node):	
  NOT(child(x,x));
	
% when y is the parent of x,
% then y cannot be a child of x.  
ASSERT FORALL (x, y: Node):
  child(x,y) => NOT(child(y,x));
  
% a child has only 1 parent.
ASSERT FORALL (x, y, z: Node):
  (child(x,y) AND DISTINCT(y,z)) 
    => NOT(child(x,z));

% x is the j-th child of y.
children: (Node, Node, INT) -> BOOLEAN;
ASSERT FORALL (x,y:Node, j:INT): 
  children(x, y, j) => 
    child(x, y) AND j >= 0;

% child position/index starts at 0.
firstChild:		(Node, Node) -> BOOLEAN;
ASSERT FORALL (x, y:Node):
  firstChild(x, y) <=> 
    children(x, y, 0);

% every other child must have an index 
% smaller than that of the last child.
lastChild: (Node, Node) -> BOOLEAN;	
ASSERT FORALL (x, y:Node): 	
	lastChild(x, y) => EXISTS(j:INT): 
	  children(x, y, j) AND 
	  (FORCALL(z:Node, k:INT): 
	    (children(z, y, k) AND 
	    DISTINCT(z, x)) => k < j);

% children.length equals to 1 plus
% the child index of the last child.
childrenLength:	(Node) -> INT;
ASSERT FORALL (y:Node, j:INT):
	childrenLength(y) = j <=> 
	  EXISTS(x:Node): (lastChild(x, y) 
	    AND children(x, y, j-1));

% example of inversion
% y is the parent of x, is the same as 
% x is a child of y.
parent: (Node, Node) -> BOOLEAN; 
ASSERT FORALL (x, y: Node): 
  parent(y, x) <=> child(y, x); 
\end{lstlisting} 
\end{figure}

\header{Negation.}
Negations are supported by SMT solvers by default.  To negate a constraint we wrap the constraint with {\tt NOT()}.
%While it is possible to define a customized data structure to mimic the DOM tree in CVC, the boolean/integer function approach is simple and convenient for negating conditions and going into different execution paths.
%Because Web browsers also support accessing the DOM tree via XPath~\cite{documentEvaluate}, we defined additional SMT functions: {\tt following\_sibling}, {\tt preceding\_sibling}, {\tt descendant} and {\tt ancestor}.

\header{Solver Output into HTML.}
Because we used Boolean functions and Integer functions, CVC is only going to solve for the interdependent logic constraints and it does not directly yield a concrete DOM tree.  
Instead, CVC only expands the quantifiers and it outputs more {\tt ASSERT} statements for making the constraints satisfiable.  

In Sample Code \ref{domOr}, the condition inside the first {\tt if} statement has 2 sub conditions: 
\begin{compactitem}
\item {\tt d === elem.firstElementChild} ({\tt line 3}), and 
\item {\tt d === b.lastElementChild} ({\tt line 4}).
\end{compactitem}
When the DOM solver solves this {\tt if} statement for going to the {\tt true} branch, it would output {\tt "ASSERT lastChild(d, b);"} because it has decided on making the second sub condition ({\tt line 4}) {\tt true}.

To build the DOM tree, we built an API that parses the CVC output text and builds a model for the DOM.  
In the CVC output, CVC creates many temporary variable names, thus each DOM element can have more than one alias.  
To consolidate the aliases, we start with DOM operations expressing the parent child relationship because each child can have only 1 parent.  
We also used other deterministic DOM operations such as {\tt firstChild()} and {\tt lastChild()} to group aliases together.  
For example, if {\tt x} is the first child of {\tt y} and {\tt z} is also the first child of {\tt y}, then {\tt x} and {\tt z} are two aliases of the same DOM element.  

Once the parent child relationship is established, our next step is to organize the ordering of children.  
Some DOM children have their positions explicit (e.g. {\tt firstChild(x, y)}, {\tt lastChild(x, y)} and {\tt children(x, y, j)}), yet very often the child parent relationship is implied.  For example, {\tt nextSibling(x, z)} implies {\tt x} and {\tt z} share the same parent.  
To calculate the ordering of children, we use the explicitly positioned children as anchors and we relate the anchors to other children by the sibling operations.  

%In case of any uncertainty, we would query the CVC model.  We have to do a binary search because querying the CVC model supports only {\tt true} or {\tt false} answers.  
%Because CVC does not support strings natively, for now we map tag names into integers in CVC.  <span> is the default tag type because <span> is an inline element and not a block.
%Once the DOM tree is defined, we search for the root of the DOM tree (the element without a parent) and generate HTML.  

% \header{DOM mutations.}

% \header{Conditional Slicing.}

\section{Implementation}
\label{impl}
\header{Concolic Driver.}
The concolic driver automates and coordinates the concolic process end to end from deducing constraints to generating HTML and driving execution.  Specifically, the concolic driver would repeatedly
\begin {compactitem}
\item Open the target URL in the Web browser
\item Load the generated html (initially empty html)
\item Execute the target JavaScript code
\item Measure coverage, and decide which path to go next
\item Call the MapDeducer, which returns the DOM constraints in text
\item Send the constraints to the DOM solver, which returns a satisfiable DOM tree
\item Go back to the first step with the newly generated HTML.
\end{compactitem}
The concolic driver would iterate the cycle until it has reached a certain point, which can be configured to be a specific number of iterations (e.g. 1,000 DOM trees generated), a certain level of coverage (e.g. 100\% branch coverage), or both. 


\header{Instrumenting and Executing JavaScript.}  
Our approach has to be generic, transparent and browser-independent.  
We use Selenium's WebDriver \cite{webdriverjs} to drive Web browsers for executing JavaScript.  
WebDriver runs on multiple browsers, thus using WebDriver meets our design goal of being browser-independent.  

When JavaScript is getting downloaded onto a Web browser, we use the WebScarab proxy \cite{webscarab} to intercept the download and to instrument code.  
The proxy passes intercepted JavaScript code to the Google Closure Compiler API \cite{ClosureCompiler}, which transforms JavaScript into calls to the operator functions. 

Both the Backward Slicer and MapDeducer are implemented as JavaScript APIs, and WebDriver natively supports calling JavaScript functions within the browser.  
Thus our approach is entirely transparent and can be applied to multiple brands of Web browsers.    


\header{Inline JavaScript \& {\tt eval()}.}  
In addition to source files, JavaScript code can also be found within {\tt eval()} and inlined as attributes of a DOM element inside the HTML declaration (e.g. {\tt <body onload="runFunction()">}).
We instrument each {\tt eval(code)} statement into {\tt eval(instrument(code))}.  We do not override the native {\tt eval()}, because the native {\tt eval()} must be called inside the closure to give {\tt code} access to the closure's local variables.  
The {\tt instrument()} function would send the {\tt code} to the proxy for instrumentation via a XMLHttpRequest.  

To instrument inline JavaScript, we traverse the webpage's original existing HTML using the JSoup API~\cite{jsoup}.  
Once we detect DOM attributes that contain JavaScript, we pass the JavaScript to the Google Closure Compiler API for instrumentation.  
For newly created elements, e.g. {\tt elem.innerHTML = text}, we use getters and setters to detect the new elements.  
Once detected, we traverse the new elements, extract JavaScript and call the {\tt instrument()} function.  
% Tudu


\header{DOM Solver.}  
CVC allows writing the constraints in Java, yet we do not want to hardcode the constraints because the constraints are different for each execution path.  
Thus we use Java's ProcessBuilder~\cite{processbuilder} class to communicate with the executable (.exe) version of CVC.  We decided to use CVC3 \cite{cvc3} rather than CVC4~\cite{cvc4} because CVC3 is generally more stable during our experimentation.  
The API that parses the CVC output is also implemented in Java, thus we used the W3C DOM API~\cite{DomAPI} for building a satisfiable DOM tree and generating HTML.  
%QUnit provides a <div> called the fixture which can be set as the HTML for a test case.   Before execution, WebDriver would set the {\tt innerHTML} of the fixture element to the generated HTML.  


\header{Limited Path Coverage.}  
% zero, 1 and n.
Very often we would not know how many times to execute a loop.  For example, in Sample Code ~\ref{dom0}, there is no upper limit to the number of children that {\tt field} can have.  
\tool would execute loops zero, one, and {\tt n} times, where {\tt n} can be configured for a particular loop or for all loops.  Thus \tool would achieve limited path coverage rather than full path coverage.  


%\header{Indexing Functions.}  
%Most of the time JavaScript functions are defined inside closures and are not accessible~\cite{privatefunctions}.  
%When we instrument JavaScript code, we extract the functions by assigning them to an object that we have access to.  
%Because functions can have the same name, we use the node number in the JavaScript Abstract Syntax Tree as ID.

%\header{QUnit}
%\tool has integration with QUnit~\cite{qunit} so that existing test suites can automatically take advantage of \tool without additional manual effort.  
% \tool is also extensible to be integrated with other test frameworks~\cite{jstests}.
% Thus given a test case, be its inputs were generated manually or automatically, \tool can be used to help the test case and its assertions get fully utilized.

\header{Inserting HTML into Webpage.}
In HTML, certain DOM elements must be wrapped inside other DOM elements~\cite{jsninja}.  

For example, table cells ({\tt <td>} and {\tt <th>}) have to be contained inside a table row ({\tt <tr><td></td></tr>}).
Table rows ({\tt <tr>}) must be contained inside 
\begin{compactitem}
\item a table header: {\tt <thead> <tr></tr>... </thead>}, 
\item a table body: {\tt <tbody> <tr></tr>... </tbody>}, or
\item a table footer: {\tt <tfoot> <tr></tr>... </tfoot>}.
\end{compactitem}
\sloppy{
Similarly, every table header, table body, and table footer must be contained inside a table too: 
{\tt <table>} {\tt <thead>...</thead>} {\tt <tbody>...</tbody>} {\tt <tfoot>...</tfoot>} {\tt </table>}.
Each table contains at most 1 header, 1 body and 1 footer.  
}

{\tt <option>} and {\tt <optgroup>} have to be inside a {\tt <select multiple="multiple"> ... </select>}.
{\tt <legend>} has to be inside a {\tt <fieldset> ... </fieldset>}.
\tool processes the generated HTML before inserting it into a webpage.  

\section{Evalulation}


\subsection{Research Questions.}

\header{Quality vs. Time}
Our first set of research questions focuses on optimization quality vs. time.
Given more time, any gradient method yields a better optimization.  
Our focus here is to identify which gradient method is the most suitable for data-scientists prototyping recommender systems.
In terms of suitability, we mean the gradient method that yields the best quality optimization within the shortest amount of time.
Here, we consider the general \emph{SAG} approach \emph{as is}. 
The next set of research questions studies the specific \emph{space vs. time} trade-off between \tool and the na$\ddot{i}ve$ approach to \emph{SAG}.

Between \emph{SAG}, full deterministic gradient and stochastic gradient,
\begin{sloppy}
\begin{compactenum}
\item Which gradient method yields a better optimization given the same amount of time?
\item Which gradient method uses the shortest amount of time to reach a similar quality of optimization?
\item Can \emph{SAG} and specifically \tool work well with different objective functions in recommender systems?
\item Can \emph{SAG} and specifically \tool work well with different matrix datasets?
\end{compactenum}
\end{sloppy}


\header{Space vs. Time}
Our second set of research questions investigates whether re-computing is worth the additional time.
Here, we investigate the actual space vs. time trade-off between \tool vs. the na$\ddot{i}$ve approach to \emph{SAG}:
Compared to the na$\ddot{i}ve$ approach to SAG, in practice
\begin{sloppy}
\begin{compactitem}
\item How much slower is \tool due to re-computing?
\item How much memory does \tool save?
\end{compactitem}
\end{sloppy}



\header{Objective Functions.}
% cite objective functions



\header{Datasets}
% cite datasets



\header{Experiment Setup}
% a table?
%\section{Future Work}  

\header {Solving DOM Attributes.}
Most DOM Attributes take the form of integers and strings.  

\header{Element vs. Node.}
% inheritance in CVC

\header{DOM Mutations.}
% conditional slicing

\header{Multiple Data Types.}

\header{Designer-Developer Collaboration.}
The majority of JavaScript bugs are DOM related~\cite{frolin2013}.
Ultimately, our higher level goal is to foster closer collaboration among designers and developers.\footnote{As part of separating concerns, design and development are often done by distinct individuals having very different backgrounds.}
For example, because \tool generates reference HTML for satisfying code execution, it can be used to detect DOM mismatches between the designer's HTML vs. what is expected in the developer's JavaScript code.  
A mismatch may not always be the developer's fault.  Sometimes it is possible that the designer may have made a mistake when updating the HTML or may be just too busy and have forgotten to notify the developer about a change.  

%XML, other programming languages 

% how about HTML5 canvas

\header{Other SMT Solvers.}
Microsoft Z3~\cite{z3}.  SMT-lib language.







\section{Related Work}
%Whether the test inputs are manually, randomly (e.g. ~\cite{artemis}) or symbolically (e.g. ~\cite{kudzu, jalangi}) defined.


\header{Concolic Testing.}
% cute, jalangi, kudzu, eventConcolic
%Having the constraints generated from a backward slice, concolic testing would use a constraint solver to generate input that would eventually drive the execution of each condition towards a specific branch.  
While there have been numerous publications on concolic testing (e.g. ~\cite{cute, klee, eventConcolic}), 
Kudzu~\cite{kudzu} and Jalangi~\cite{jalangi} are the only two for concolic testing software that are written in a dynamically typed language; and both focus on JavaScript.  

A main contribution of Kudzu is a string solver that can handle select regular expressions.  
The solver is deployed to generate strings as UI inputs for detecting security vulnerabilities in JavaScript Web applications.   
A main contribution of \tool is our DOM solver.  Our solver generates HTML for running and testing JavaScript code that contains DOM operations.    
While DOM trees are usually represented in HTML string form, designing a DOM solver is different from designing a regex string solver.
As discussed in the Challenges section, the main reasons are that the DOM solver has to support a 2D hierarchical tree structure while strings are usually single dimensional.  
Moreover, inferring implicit clues from DOM operations is also different from inferring regex patterns.  
In addition, the architecture of \tool is implemented to run on multiple Web browsers, while Kudzu runs only on their single proprietary browser that supports Kudzu's component~\cite{flax} for tracing and slicing.    

A main contribution of Jalangi is their system of shadow values for selective record and replay.  
In the shadow system, they encapsulate each data value into an object; the encapsulated object can contain any metadata (the shadow) about the actual data value.  
While our system of decorated execution is similar to Jalangi's shadow system, a tester using Jalangi would manually identify which variables are inputs and would manually specify each input's type.  
Jalangi then generates various input values for those variables that are manually identified by the tester.  
In contrast, \tool uses post-order tree traversal for automatically identifying possible inputs.  

When \tool instruments JavaScript, it would label each constant value as a constant.
For example, the JavaScript statement {\tt var x = "string";} would be instrumented into {\tt var x = \_CONST("string");}.  
  
Therefore, during the post order traversal, if a tree leaf of the decorated objects tree is not labelled, the variable inside the leaf would be identified as a candidate input, 
because the data value does not come from within the code.  
\tool generates a dynamic backward slice and thus executes the code, therefore the data type of candidate inputs can easily be determined from the actual data value.  

Another differentiation is that for supporting the DOM we designed a Trace MapDeducer for extracting and transforming code-centric backward slice into DOM-centric constraints for the DOM solver.  
Both CUTE~\cite{cute} and Jalangi use the CVC3~\cite{cvc3} solver for supporting integers and strings.  

%jalangi and Kudzu focuses on generating inputs, and did not address the problems of closures and making test cases runnable.
% Whether these test cases are generated manually, randomly (e.g. ~\cite{artemis}) or symbolically (e.g. ~\cite{kudzu, jalangi}), these test cases cannot be properly run or asserted because 


% may move to Intro
%\header{HTML Generation.}
%\header{Emmet or Zen coding.}
%Emmet (formerly Zen coding) \cite{ZenSjeiti, ZenCoding} generates DOM elements as output.  
%Input to Emmet is the abbreviation of a CSS query that precisely specifies the DOM structure in a declarative manner.
%A major difference is that \tool solves for complex logic constraints (e.g. AND, OR conditions).
%Pythia uses the Web app's existing HTML.


\header{Constraint Solvers.} 
% xml solver, cvc, z3
Constraint solvers (e.g. SAT solvers) solve for parameters that satisfy a set of predefined constraints.  

Genev\`{e}s et al. developed an XML solver~\cite{xmlsolver} that takes limited XPaths as inputs; then it outputs XML that would satisfy those XPath conditions. 
Initially we intended to extending the XML solver.  However, after experimentations, we find it difficult to encode DOM node attributes into the XML solver and 
more importantly the XML solver is not scalable to more than 5 unique nodes. 

CVC~\cite{cvc3, cvc4} is a general purpose SMT solver and it is more scalable.  
However, being a general purpose solver also means that CVC does not natively support the tree structure defined in the DOM API.  
Our DOM solver uses quantifiers to encode and model the DOM within CVC.  
Nevertheless, the output of CVC yields only a description of the desired DOM tree (e.g. node A is child of node B), rather than the actual XML/HTML. 
Thus we have to take additional steps to transform CVC outputs into HTML.


\header{Feedback Directed Testing.}   
Feedback Directed Testing is an adaptive testing approach that uses the outcome of executing an input, to determine what the next input should be for achieving a goal, mostly maximizing coverage.  
Random testing and concolic testing are two major formats of feedback directed testing that is automated.  
Concolic testing is a form of feedback directed testing because it conducts backward slicing to generate inputs, and then it uses the resulting executed path as feedback for generating new inputs.  

In random testing for JavaScript Web applications, Artemis~\cite{artemis} randomly generates initial inputs and uses the output of functions (rather than executed paths) as feedback, for increasing coverage.  
Pythia~\cite{pythia} also generates initial inputs randomly, their feedback is changes to a state flow graph model, and their goal is to maximize the number of functions being called.  
The number of functions being called is directly proportional to the number of lines covered.  
For covering JavaScript code that contains DOM operations, Pythia would use a Web application's existing HTML if such HTML is available.  
Thus Pythia does not generate new HTML for covering execution paths that conflict with the existing HTML.  
For example, when the {\tt true} branch of an {\tt if} statement requires a DOM element having 10 children, 
Pythia would never enter the {\tt true} branch if the existing HTML does not contain 10 children for that DOM element.

%In contrast to Artemis and our tool, Pythia is for regression testing; it requires a previous version of bug free software, and also mandates that the current version has zero change in both behavior and interface such that the same input always yields identical output. 
%When a software requires regression testing, it either has a bug fixed (violates the bug-free requirement) or has an improvement or a new feature implemented (may violate the zero change requirement).  
%An occasion when both external behavior and interface do not change, is when a function's internal implementation has changed for improving only performance. 
%Then, JavaScript applications are known to lack determinism~\cite{mugshot}, meaning the same source code is known to yield different outputs even for the same inputs.  
%Moreover, while we aim to infer an HTML input, Pythia uses the application's existing HTML to unit test JavaScript functions.

% \footnote{Another category of dependencies is closure variables.}.

\section{Future Work \& Conclusion}
This paper is the first in the series of our study on data scientists prototyping model-based recommender systems.
We explored the convex-optimization perspective of the problem; and we propose Stochastic Average Gradient as a viable alternative to Full Deterministic Gradient and Stochastic Gradient.
In theory, we proved that our extension and adaptation of SAG has asymptotic time complexity and asymptotic space complexity as efficient as stochastic gradient.
In practice, through extensive evaluation we demonstrated that, even before any fine-tuning or optimization of the implementation, our SAG-MF algorithm still outperforms both Full Deterministic Gradient and Stochastic Gradient in terms of convergence.
% the recommender perspective 
In the future, we aim to complete our ongoing work on the metrics perspective and on the software engineering perspective.  
Given a dataset, the quality of a recommender system is often evaluated in various metrics: e.g. precision, recall, area under curve, reciprocal rank, NDCG, and variants of the above such as top-K precision and top-K hit rate.
% what is NDCG?
Many papers in the literature claim their objective function is better by illustrating that their objective function performs in some of these metrics better than other objective functions.  
Therefore, in the metrics perspective, we are exploring and investigating which factors are more relevant and important to scoring high in the various metrics: is it the objective function, is it the method for convex-optimization such as SAG, is it other fine-tuning mechanisms such as bootstrapping, or is it the hyper parameters that we use in convex-optimization but which can also be specific to the dataset?
% the software engineering perspective 
In the software engineering perspective, we are studying how to increase the productivity of data scientists.  At this point, we are designing and developing a mix-n-match and plug-n-play framework that enables data scientists would make it quick to prototype and experiment different combinations objective functions, datasets, gradient methods, hyper parameters and evaluation metrics.  


\bibliographystyle{abbrv}  
\bibliography{refs}

\end{document}
