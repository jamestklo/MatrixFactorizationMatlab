\documentclass{sig-alternate}
\pdfpagewidth=8.5in
\pdfpageheight=11in
\usepackage[utf8]{inputenc}
\usepackage[T1]{fontenc}
\usepackage[font=bf]{caption}
\usepackage{listings}
\lstset{language=Python}
\usepackage{color}
\usepackage{xspace}
\usepackage{hyperref}
\usepackage{url,moreverb,graphicx}
\usepackage{ifthen}
\usepackage{paralist}
\usepackage{lipsum}
\usepackage{tikz}
\usetikzlibrary{patterns}
\usepackage{pgfplots}
\usepackage{comment}
\usepackage{amsmath}

\clubpenalty=10000
\widowpenalty = 10000

\definecolor{lightgray}{rgb}{.9,.9,.9}
\definecolor{darkgray}{rgb}{.4,.4,.4}
\definecolor{purple}{rgb}{0.65, 0.12, 0.82}
\lstdefinelanguage{JavaScript}{
  keywords={typeof, new, true, false, catch, function, return, null, catch, switch, var, for, if, in, while, do, else, case, break},
  keywordstyle=\color{blue}\bfseries,
  ndkeywords={class, export, boolean, throw, implements, import, this, assert, forall, exists, not, int},
  ndkeywordstyle=\color{darkgray}\bfseries,
  identifierstyle=\color{black},
  sensitive=false,
  comment=[l]{//},
  morecomment=[s]{/*}{*/},
  morecomment=[l]{\%},
  commentstyle=\color{purple}\ttfamily,
  stringstyle=\color{red}\ttfamily,
  morestring=[b]',
  morestring=[b]"
}

\lstset{
   language=JavaScript,
   backgroundcolor=\color{lightgray},
   extendedchars=true,
   basicstyle=\scriptsize\ttfamily,
   showstringspaces=false,
   showspaces=false,
   numbers=left,
   numberstyle=\scriptsize,
   numbersep=9pt,
   tabsize=2,
   breaklines=true,
   showtabs=false,
   captionpos=b,
   xleftmargin=4.0ex
}

\newcommand{\tool}{\textsc{SAG-re}\xspace}

\newcommand{\code}[1]{{\texttt{#1}}}
\newcommand{\js}{{Java\-Script}\xspace}\newcommand{\header}[1]{\par\smallskip\noindent\textbf{#1}}

\newboolean{showcomments}
\setboolean{showcomments}{true}
\ifthenelse{\boolean{showcomments}}
{\newcommand{\nb}[2]{
\fbox{\bfseries\sffamily\scriptsize#1}
{\sf\small$\blacktriangleright$\textit{#2}$\blacktriangleleft$}
}
}
{\newcommand{\nb}[2]{}
}
\newcommand\james[1]{\nb{James}{#1}}
\newcommand\eric[1]{\nb{Eric}{#1}}
\newcommand\ali[1]{\nb{Ali}{#1}}

\renewcommand*{\lstlistingname}{Listing}

\newcounter{head}
\newcommand{\challenge}[1]{%
  \refstepcounter{head}
  \header{\thehead.\ \ #1.}
%\newcommand{\head}[1]{\textbf{#1.}\ \ }
}


\begin{document}

% --- Author Metadata here ---
%\conferenceinfo{ISSTA 2014, Jul 21-26, 2014}{San Jose, California}
%\CopyrightYear{2014} % Allows default copyright year (20XX) to be over-ridden - IF NEED BE.
%\crdata{0-12345-67-8/90/01}  % Allows default copyright data (0-89791-88-6/97/05) to be over-ridden - IF NEED BE.
% --- End of Author Metadata ---


%\title{Prototyping Recommender Systems using {\ttlit SAG-recomputed}}
\title{{\ttlit SAG-re}: Faster Prototyping of Recommender Systems using Stochastic Average Gradient}
\numberofauthors{3} 
\author{
\alignauthor
James Lo\\	
       \affaddr{Computer Science}\\
       \affaddr{University of British Columbia}\\	   
       \affaddr{Vancouver, Canada}\\
       \email{tklo@cs.ubc.ca}
}
%\date{24 January 2014}

\maketitle
\begin{abstract}
\end{abstract}

% A category with only the three required fields
\category{H.3.3}{Information Storage and Retrieval}{Information Search and Retrieval}[Information Filtering]
%A category including the fourth, optional field follows...
%\category{D.3.2}{Software}{Programming Languages}[JavaScript]

\terms{Asymptotic Time Complexity, Asymptotic Space Complexity, Experimentation}

\keywords{Recommender systems, collaborative filtering, matrix factorization, stochastic gradient}

\section{Introduction}
% Why RecSys
Shopping, text or display advertising, renting movies, listening to music… recommender systems are prevalent and ubiquitous in our daily lives.  
% Why Matrix Factorization  
Matrix factorization (\emph{MF}) is a popular technique in model-based recommender systems.  
Indeed, MF has been utilized extensively in past research for handling both explicit \cite{mmmf2005fast, mnar, gapfm} ratings, 
and implicit \cite{wrmf2008hu, wrmf2008pan, climf, bpr, mnar} feedback.  

% Pitfalls with FG and SG  
In recommender systems that utilize matrix factorization, most optimize an objective function.  
Full deterministic gradient \emph{FG} and stochastic gradient \emph{SG} are the two main gradient methods for optimization.  
All of the recommender systems that we mentioned utilize one of \emph{FG} or \emph{SG}.  
However, both full deterministic gradient and stochastic gradient have pitfalls when it comes to prototyping recommender systems.  

Full deterministic gradient can offer high quality optimizations.  
However, FG is slow because in each iteration of optimization, FG has to pass through all the samples in the dataset.  
Stochastic gradient is relatively fast; iteration cost is low because each iteration of SG looks at only one or a few samples.  
However, the trade-off with SG is that it often provides relatively low quality optimizations.  

High quality optimizations with a low iteration cost is important when building recommender systems.  
The first reason is that data scientists often have to run repeated experiments: e.g. with different objective functions, different metrics, different datasets, and different optimization parameters.  
The second reason is that product life cycles are shortening in the age of agile software engineering.  
Thus data scientists are often faced with the challenge of running more experiments and producing high quality results with less time.  

%Why SAG in prototyping model-based recommender systems  
In this paper, we study the challenge from the perspective of convex-optimization.  
We propose using the stochastic average gradient \emph{SAG} method \cite{schmidt2013minimizing, roux2012stochastic} as a viable alternative to using \emph{FG} and \emph{SG} during the prototyping process.  
\emph{SAG} has the distinctive advantage that its optimization quality is proven to be much better than \emph{SG}; at the same time \emph{SAG}'s iteration cost is asymptotically identical to \emph{SG}.  
However, adapting \emph{SAG} to matrix factorization is difficult because \emph{SAG} requires previously-computed gradients and storing these gradients can lead to very high asymptotically space complexity.  
We explore the challenge with space-complexity, and addresses it by proposing a re-computation approach \tool that re-computes the previously-computed gradients.  
Our re-computation approach preserves the low iteration cost of \emph{SAG}.  
The asymptotic space complexity of our \tool approach is as efficient as the memory-less full deterministic gradient, and stochastic gradient.  

To the best of our knowledge, we are the first to
\begin{compactitem}
\item Identify the pitfalls associated with using full deterministic gradient and stochastic gradient when data-scientists prototype model-based recommender systems.  
\item Propose Stochastic Average Gradient (\emph{SAG}) as a viable alternative for yielding faster convergence in the convex-optimization of matrix factorization.  
\item Extend SAG into \tool for matrix factorization, resolve the space complexity challenge in adapting SAG from the domain of large-scale supervised-machine-learning into the domain of prototyping recommender objective-functions.  
\item Prove in theory, that our \tool approach has identical asymptotic time complexity and identical asymptotic space complexity as stochastic gradient.  
\item Extensively evaluate and compare SAG-MF across multiple RecSys objective functions and diverse datasets.  
\item Demonstrate in practice that, even before any optimization of the implementation, our SAG-MF algorithm still yields faster convergence despite the additional time of re-computation, and that SAG-MF uses memory at a level similar to stochastic gradient.  
\end {compactitem}
	
\section{Background and Terminology}
To motivate our paper and the space complexity challenge, we first introduce the background and the terminology that we use.  


\header{Matrix Factorization.}
Model-based recommender systems approximate the \emph{user-item} matrix \emph{A} through 
the dot-product of the \emph{user}-matrix \emph{U} and the \emph{item}-matrix \emph{V}: $\hat{A} = U * V.$  

The \emph{user-item} matrix \emph{A} is a \emph{nRows}-by-\emph{nCols} matrix.
\emph{A} can be \emph{sparse}; thus we use \emph{N} to indicate the number of non-zero entries in \emph{A}.

The \emph{approximation} matrix $\hat{A}$ also has \emph{nRows} rows, and \emph{nCols} columns.  
$\hat{A}$ is not a sparse matrix.  The goal of model-based recommendation is to use the non-zero entries to approximate the missing entries in \emph{A}. 
When multiplying \emph{U} and \emph{V}, the latent dimensions \emph{nDims} cancels-out in the dot product. 
This is why the \emph{approximation} matrix has identical dimensions as the original \emph{user-item} matrix.  

The \emph{user} matrix \emph{U} is \emph{nRows}-by-\emph{nDims}: \emph{U} has \emph{nRows} rows, and \emph{nDims} columns.  
\emph{nDims} is the number of latent dimensions.  
The \emph{item} matrix \emph{V} is \emph{nDims}-by-\emph{nCols}.  


\header{Optimizing an Objective Function.}
The goal of matrix factorization is to find the best \emph{U} and the best \emph{V} whose dot product optimizes an objective function:
\begin{equation} \label{eq:argmin}
\operatorname*{arg\,min (or\,arg\,max)}_{U,V} \left[ f(U, V) = \sum_{i=1}^{nRows} \sum_{j=1}^{nCols} f(\bar{u}_{i}, \bar{v}_{j}) \right]
\end{equation}

When we take the gradient of the objective function with respect to a row in the \emph{user} matrix \emph{U} (e.g. $\bar{u}_{i}$), we sum up the gradient of all the entries in $\hat{A}$ that belong to the same row $\bar{u}_{i}$.  
\begin{equation} \label{eq:gradU}
\frac{\text{d}f(U,V)}{\text{d}\bar{u}_i} = \sum_{j=1}^{nCols} \frac{\text{d}f(\bar{u}_i,\bar{v}_j)}{\text{d}\bar{u}_i}
\end{equation}

Similarly, when we take the gradient with respect to a column of \emph{V} (e.g. $\bar{v}_{j}$), we sum up the gradients across different rows that belong to the same column: 
\begin{equation} \label{eq:gradV}
\frac{\text{d}f(U,V)}{\text{d}\bar{v}_j} = \sum_{i=1}^{nRows} \frac{\text{d}f(\bar{u}_i,\bar{v}_j)}{\text{d}\bar{v}_j}
\end{equation}

Both $\frac{\text{d}f(\bar{u}_i,\bar{v}_j)}{\text{d}\bar{u}_i}$ and $\frac{\text{d}f(\bar{u}_i,\bar{v}_j)}{\text{d}\bar{v}_j}$ are vectors of length \emph{nDims}, the number of latent dimensions.
Specifically, $\frac{\text{d}f(\bar{u}_i,\bar{v}_j)}{\text{d}\bar{u}_i}$ is a \emph{1}-by-\emph{nDims} row vector;
$\frac{\text{d}f(\bar{u}_i,\bar{v}_j)}{\text{d}\bar{v}_j}$ is a \emph{nDims}-by-\emph{1} column vector.
Similarly, the summed-up gradient $\frac{\text{d}f(U,V)}{\text{d}\bar{u}_i}$ is a row vector, and $\frac{\text{d}f(U,V)}{\text{d}\bar{v}_j}$ is a column vector, of length \emph{nDims}.

In \emph{SAG}, storing only the summed-up gradients is not sufficient for matrix factorization.
The reason is that, each iteration of \emph{SAG} requires the fine-grain gradients of individual entries 
(e.g. $\frac{\text{d}f(\bar{u}_i,\bar{v}_j)}{\text{d}\bar{u}_i}$ and $\frac{\text{d}f(\bar{u}_i,\bar{v}_j)}{\text{d}\bar{v}_j}$) 
that we previously sampled at an iteration before \emph{t}.  
As we will prove, when directly applied to matrix factorization without using our \tool approach, \emph{SAG} will have a asymptotic space complexity of $\theta$(\emph{nDims}*(min(\emph{M},\emph{N})+\emph{nRows}+\emph{nCols})).
\emph{M} is the number of \emph{distinct} samples we have previously visited.  
Usually, $\emph{M} = B*t$, where $B$ is the batch size per iteration, and $t$ is the number of iterations previously done.  

Here, we want to point out that \tool preserves the low asymptotic time complexity as \emph{SAG}; and \tool reduces asymptotic space complexity to $\theta$(\emph{nDims}*(\emph{nRows}+\emph{nCols})+\emph{N}). 
We will prove that this asymptotic space complexity is as compact as any memory-less approach.


\header{Gradient Methods in Matrix Factorization.}
Gradient methods are iterative methods of optimization.  
When we increase the number of iterations, we expect the quality of optimization to also increase over time.
At each iteration, gradient methods sample a batch of $B$ entries, calculate the gradients of these entries, and use the calculated gradients to update $U$ and $V$ for the next iteration:
\begin{equation} \label{eq:Ut}
U^{t+1} = U^t + \frac{\alpha^{t}}{B}\left(\sum_{b=1}^{B}\frac{\text{d}f(\bar{u}_{entry(b).i}, \bar{v}_{entry(b).j})}{\text{d}\bar{u}_{entry(b).i}}\right)
\end{equation}
\begin{equation} \label{eq:Vt}
V^{t+1} = V^t + \frac{\alpha^{t}}{B}\left(\sum_{b=1}^{B}\frac{\text{d}f(\bar{u}_{entry(b).i}, \bar{v}_{entry(b).j})}{\text{d}\bar{v}_{entry(b).j}}\right)
\end{equation}

At iteration \emph{t}, $U^{t}$ is the current approximation of $U$.
We use the gradients of the sampled batch of entries to update $U^{t}$ into $U^{t+1}$ for iteration \emph{t+1}.

$\alpha^{t}$ is the \emph{learning-rate} or \emph{step-size}, at iteration \emph{t}.  
When the goal of our optimization is to maximize an objective function, we apply \emph{gradient-ascent} on \emph{U} and \emph{V}; thus we set $\alpha^{t} > 0$.
When we try to minimize an objective function, we apply \emph{gradient-descent} and set $\alpha^{t} < 0$.

\emph{entry(b)} is the \emph{b}-th entry in our batch of samples.  
\emph{entry(b).i} is the \emph{row} number of the entry; \emph{entry(b).j} is the \emph{column} number of the entry sampled from $A$.

Full deterministic gradient (\emph{FG}) takes all $N$ samples at each iteration;  $B$ = $N$ in \emph{FG}.
Stochastic gradient (\emph{SG}) takes only one or a few samples per iteration: $B$ is usually a constant much less than $N$.


\header{Stochastic Average Gradient.}
Stochastic Average Gradient (\emph{SAG}) requires a memory of previously-computed gradients: e.g. $\bar{m}_{U}^{t}$ and $\bar{m}_{V}^{t}$ for matrix factorization.
Each iteration of \emph{SAG} uses a sampled batch of entries to update the memory.
After the update, \emph{SAG} then applies the updated memory $\bar{m}_{U}^{t+1}$ and $\bar{m}_{V}^{t+1}$ respectively on calculating $U^{t+1}$ and $V^{t+1}$:
\begin{equation} \label{eq:sag_mi}
  \bar{m}_{entry(b).i}^{t+1} = \frac{\text{d}f(\bar{u}_{entry(b).i}, \bar{v}_{entry(b).j})}{\text{d}\bar{u}_{entry(b).i}}
\end{equation}
\begin{equation} \label{eq:sag_mu}
  \bar{m}_{U}^{t+1} = \bar{m}_{U}^{t} + \sum_{b=1}^{B}\left[\bar{m}_{entry(b).i}^{t+1} - \bar{m}_{entry(b).i}^{t}\right]
\end{equation}
\begin{equation} \label{eq:sag_ut}
  U^{t+1} = U^{t} + \frac{\alpha^{t}}{B}\left(\bar{m}_{U}^{t+1}\right)
\end{equation}
\begin{equation} \label{eq:sag_mj}
  \bar{m}_{entry(b).j}^{t+1} = \frac{\text{d}f(\bar{u}_{entry(b).i}, \bar{v}_{entry(b).j})}{\text{d}\bar{v}_{entry(b).j}}
\end{equation}
\begin{equation} \label{eq:sag_mv}
\bar{m}_{U}^{t+1} = \bar{m}_{U}^{t} + \sum_{b=1}^{B}\left[\bar{m}_{entry(b).j}^{t+1} - \bar{m}_{entry(b).j}^{t}\right]
\end{equation}
\begin{equation} \label{eq:sag_vt}
  V^{t+1} = V^{t} + \frac{\alpha^{t}}{B}\left(\bar{m}_{V}^{t+1}\right)
\end{equation}

%$\bar{m}_{entry(b).i}^{t}$ and $\bar{m}_{entry(b).j}^{t}$ are the memory of previously-computed gradients.  
We apply \emph{SAG} into matrix factorization for two reasons.  
First, \emph{SAG} has iteration cost as low as stochastic gradient (\emph{SG}).
Second, \emph{SAG}'s convergence rate is faster than \emph{SG}, and sometimes as fast as full deterministic gradient (\emph{FG}).

\header{Convergence rate, Iteration cost, and Prototyping recommender systems.}
At a high level, the ideal combination of a fast convergence rate and a low iteration cost implies a better optimization in a shorter amount of time when data-scientists prototype model-based recommender systems.
An intuition behind gradient methods is that, at least for objective functions that are convex, the gradients guide the updates of $U^t$ and $V^t$ towards the direction of optimization.  
Convergence rate measures how many iterations a gradient method is expected to take towards reaching optimization. 
Iteration cost measures how many entries we sample per iteration.

Full deterministic gradient has the best possible convergence rate because each iteration of \emph{FG} samples all $N$ entries in the dataset.
However, while \emph{FG} is guaranteed to take a less number of iterations than \emph{SG} to reach optimization, sampling all $N$ entries per iteration slows down \emph{FG} overall because the optimization process would still take many iterations.
Depending on the mathematical properties of the objective function, stochastic gradient often has much slower convergence rates because \emph{SG} samples only one or a few random entries per iteration.
Therefore, while \emph{SG} has the lowest possible $\theta(1)$ iteration cost, overall \emph{SG} is still slow because \emph{SG} would take many more iterations to reach optimization.

\emph{SAG} speeds-up the convergence rate by reusing the gradients of past samples.  
Reusing past gradients enables \emph{SAG} to look at $\theta(1)$ samples per iteration and to achieve the lowest possible iteration cost.  
Our evaluation illustrates that \emph{SAG} gives a better optimization with less time than \emph{FG} and \emph{SG}.  
In this paper, we minimize the drawbacks or costs of using \emph{SAG} in matrix factorization while preserving \emph{SAG}'s benefits.
\section{Challenge}
When applying \emph{SAG} into matrix factorization, a main challenge is to make available the fine-grain gradients of previously-sampled entries: 
$\bar{m}_{entry(b).i}^{t}$ from equation \ref{eq:sag_mu}, and 
$\bar{m}_{entry(b).j}^{t}$ from equation \ref{eq:sag_mv}


%\section{approach}
Similar to the chain-rule approach, \tool re-computes $\bar{m}_{entry(b).i}^{t}$ in equation \ref{eq:sag_mu} and $\bar{m}_{entry(b).j}^{t}$ in equation \ref{eq:sag_mv}:

\begin{equation} \label{eq:sagre_mu}
  \bar{m}_{U}^{t+1} = \bar{m}_{U}^{t} + \sum_{b=1}^{B}\left[\bar{m}_{entry(b).i}^{t+1} - \frac{\text{d}f(\bar{u}_{entry(b).i}^{s}, \bar{v}_{entry(b).j}^{s})}{\text{d}\bar{u}_{entry(b).i}^{s}} \right]
\end{equation}

The main difference is that, all 

The chain-rule approach must store $min(M,N)$ different copies of past versions of $\bar{m}_{entry(b).i}^{t}$.
There are two reasons.  First, each entry can come from a different iteration.  Second, the same entry may get sampled at more than one different iterations.  
\tool resolves this issue, by predicting ahead the entires that we are going to sample.  


\emph{Theorem 5.}
\tool has convergence rate at least as fast as \emph{SAG}.
\begin{proof}

\end{proof}


\emph{Theorem 6.}
\tool has $\theta(1)$ asymptotic time-complexity and is as efficient as both \emph{SAG} and stochastic gradient.
\begin{proof}
\tool achieves the lowest possible asymptotic iteration cost.
\end{proof}


\emph{Theorem 7.}
\tool has $\theta(N + nDims*(nRows+nCols))$ asymptotic space-complexity and is as compact as any memory-less gradient method.
\begin{proof}
Indeed, \tool achieves the best possible asymptotic space-complexity (\emph{Theorem 2}).
\end{proof}

%\section{Implementation}
\header{Matlab and Mental Model.}
We implement \tool in Matlab because Matlab is a widely popular tool for prototyping algorithms in machine learning and data mining.

Matlab has an advantage that the system model of the source-code closely matches the mental model of the data-scientist.
In the eyes of a data-scientist, the close match between the system model and the mental model makes the programming-language highly usable.

For example, $\hat{A} = U*V$ is a mathematical representation for matrix multiplication.
In Matlab, the code to multiply two matrices is exactly identical to the mathematical representation above.
Thus data-scientists can exert the least amount of mental effort and seamlessly translate their thoughts into code.

In C, C++ or Java, data-scientists must deal with additional mental overhead that distracts them from concentrating on their primary goal of formulating an algorithm: 
e.g. memory allocation; pointers and references; variable type; the specific function name to use; namespaces; and the precise number, order and type of input arguments.


\header{Architecture.}
We architect our implementation so that we implement \tool only once in only one self-contained file.
Given an objective function, switching between gradient methods requires changing only one line of code.
That line of code is easily identifiable, locatable, modifiable and self-contained.
Changing that line of code also does not have any side-effects and does not require changing other lines of code.


\header{Batching and Parallelism.}
Matlab vectorizes computations and parallelizes matrix operations by default.
Matlab's parallel computing toolkit allows Matlab-users to run the same parallel version of code on a diversity of hardware from multi-core CPUs and CUDA-GPUs to multi-machine clusters. 
\tool is implemented so that at each iteration, we can calculate in parallel the fine-grain gradients of individual entries sampled in the batch.

Re-computation can also run in parallel to the computation of new gradients, 
because the re-computation of a past gradient is entirely independent from the computation of a new gradient.
To closely examine the true additional cost of re-computing, our evaluation does \emph{not} parallelize re-computing. 

%\section{Evaluation}


\subsection{Research Questions}

\header{Quality vs. Time.}
Our first set of research questions focuses on optimization quality vs. time.
Given more time, any gradient method yields a better optimization.  
Our focus here is to identify which gradient method is the most suitable for data-scientists prototyping recommender systems.
In terms of suitability, we mean the gradient method that yields the best quality optimization within the shortest amount of time.
Here, we consider the general \emph{SAG} approach \emph{as is}. 
The next set of research questions studies the specific \emph{space vs. time} trade-off between \tool and the na$\ddot{i}ve$ approach to \emph{SAG}.

Between \emph{SAG}, full deterministic gradient and stochastic gradient,
\begin{sloppy}
\begin{compactenum}
\item Which gradient method yields a better optimization given the same amount of time?
\item Which gradient method uses the shortest amount of time to reach a similar quality of optimization?
\item Can \emph{SAG} and specifically \tool work well with different objective functions in recommender systems?
\item Can \emph{SAG} and specifically \tool work well with different matrix datasets?
\end{compactenum}
\end{sloppy}


\header{Space vs. Time.}
Our second set of research questions investigates whether re-computing is worth the additional time.
Here, we investigate the actual space vs. time trade-off between \tool vs. the na$\ddot{i}$ve approach to \emph{SAG}:

Compared to the na$\ddot{i}ve$ approach to SAG, in practice
\begin{sloppy}
\begin{compactenum}
\setcounter{enumi}{4}
\item How much slower is \tool due to re-computing?
\item How much memory does \tool save?
\end{compactenum}
\end{sloppy}



\subsection{Experimental Setup}

\header{Distinct Objective Functions.}
The objective functions we choose already uses full deterministic gradient (\emph{FG}) or stochastic gradient (\emph{SG}).  
In general, any function that is differentiable, and specifically any function that uses (\emph{FG}) or (\emph{SG}) can use \emph{SAG} and \tool.
If a function is convex, then gradient methods guarantee a global optimum over time.
The functions we have chosen are distinct from each other.  The goal is to illustrate \emph{SAG} and \tool are capable of working with different objective functions.
% cite objective functions
\begin{sloppy}
\begin{compactenum}
\item \emph{Least-squares}: L2 and its variants \cite{mnar, wrmf2008hu, wrmf2008pan} are popular objective functions when building recommender systems.
\item \emph{CLiMF} \cite{climf}: Collaborative-Less-is-More-Filtering uses ordinal logistic regression to smooth the mean reciprocal rank function and to learn how a user ranks different items; 
CLiMF performs gradient ascent because the optimization goal is to maximize an objective function.
\item \emph{BPR-MF} \cite{bpr}: Bayseian Personalized Learning has an objective function that minimizes 
the difference between any two \emph{item} ratings (column entries) of the same user (same row).
BPR-MF performs gradient descent.
\end{compactenum}
\end{sloppy}


\header{Diverse Datasets.}
Our datasets are binary data that serve as implicit feedback in recommender systems. 
They represent diverse relationships including trustees \cite{epinions}, webpage bookmarking \cite{digg12month1}, casting \cite{IMDB}, social network \cite{ljournal2008}, and linking webpages \cite{wikipedia20070206}.
The datasets come from the Sparse Matrix collection at the University of Florida.
% cite datasets
\begin{sloppy}
\begin{compactenum}
\item \emph{Epinions} \cite{epinions}: $A(i,j) = 1$ when user $i$ is a trustee of user $j$, $A(i,j) = 0$ otherwise.  
The trustee relationship is not necesseary mutual.  The epinions dataset is identical to the epinions dataset that Shi et al. used in \cite{climf}. 
\item \emph{Digg12month1} \cite{digg12month1}: $A(i,j) = 1$ when user $i$ tags webpage $j$ as favorable; 0 represents no opinion. 
\item \emph{IMDB} \cite{IMDB}: $A(i,j) = 1 $ if movie $i$ has actor or actress $j$ as cast, $A(i,j) = 0$ otherwise. 
\item \emph{Live Journal} \cite{ljournal2008}:  $A(i,j) = 1 $ if user $i$ has user $j$ as his friend, $A(i,j) = 0$ otherwise. 
The graph is directed because the friendship is not neceseary mutual.
\item \emph{Wikipedia} \cite{wikipedia20070206}: $A(i,j) = 1$ if page $i$ links to page $j$, $A(i,j) = 0$ otherwise.  
\end{compactenum}
\end{sloppy}


\header{Hyper Parameters.}
For the purpose of comparison, we standardize all hyper-parameters across all objective functions, all datasets, and all gradient methods.
The only exception is that we run full deterministic gradient \emph{FG}) for only 500 iterations vs. 5,000 for stochastic gradient (\emph{SG}) and \emph{SAG}.

Convergence theory guarantees that given the same number of iterations, \emph{FG} yields a much better quality optimization than \emph{SG}.
However, our goal is to identify the gradient method that yields the best quality optimization within the shortest amount of time.
Therefore, we want to see whether \emph{FG} would take longer to yield a similar quality of optimization as \emph{SG}, and how much longer.
Through experience with our objective functions and datasets, we observed that 500 iterations \emph{FG} yields a similar quality of optimization as \emph{SG}.
As a result, we run \emph{FG} to 500 iterations, and compare how much longer 500 iterations of \emph{FG} would take than 5000 iterations of \emph{SG}.

\begin{sloppy}
\begin{compactitem}
\item Step size or learning rate: 0.0001
\item Regularization $\lambda$: 0.001; $\lambda$ is identical for regularizing both \emph{user} matrix $U$ and \emph{item} matrix $V$
\item Iterations: 5000 for \emph{SG} and \emph{SAG}, which is roughly 10\% of the number of non-zero entries in each sub-dataset.
\item Latent dimensions ($nDims$): 5 
\end{compactitem}
\end{sloppy}

For gradient descent, step size is $\alpha < 0$ and $\lambda > 0$;
for ascent, step size is $\alpha > 0$ and $\lambda < 0$.


\header{Hardware and OS.}
A MacBook Pro run all experiments that study optimization \emph{Quality vs. Time}.
Our MacBook Pro is the Late 2013 15-inch version \cite{macbookprolo}; it has OS-X Yosemite, 2.3Ghz Intel i7 quad-core CPU, 16GB RAM, and a Nvidia 750M GPU.

When studying \emph{Space vs. Time}, we measure memory usage after the first iteration.  
Initially we plan to run all experiments on the MacBook Pro.
However, the memory-profiing feature of Matlab works only on Windows.  

For the sub-datasets, we run the \emph{memory} experiment on a Dell XPS 12 \cite{dellxps12} laptop.
The Dell XPS 12 has Windows 8.1, 1.6Ghz Intel i5 dual-core CPU, 4GB RAM, and integrated graphics.  

For the full datasets, we run the \emph{memory} experiment on a remote server that has more RAM.
Our remote server has Windows Server 2008R2, 2.50Ghz Intel Xeon 2x quad-core CPUs (total 8 CPU cores) and 16GB RAM.

Both MacBook Pro and remote server have Matlab R2014a; Dell XPS 12 has Matlab R2012a.
All 3 computers have the Matlab parallel computing toolkit.



\subsection{Quality vs. Time}
%\header{Methodology.}
For the purpose of comparison, we fix the seed for generating random numbers so that 
Measure optimization quaity in each iteration. report the best optimixatiom 


%\section{Related Work}


\section{Future Work \& Conclusion}
This paper is the first in the series of our study on data scientists prototyping model-based recommender systems.  
We explored the convex-optimization perspective of the problem: we propose Stochastic Average Gradient as a viable alternative to Full Deterministic gradient and Stochastic gradient.  
By taking advantage of \emph{SAG}'s fast convergence rate and low iteration cost, we aim to enable data-scientists run more experiments and produce high quality results with less time.  
In theory, we proved that our extension and adaptation of \emph{SAG} preserves the fast convergence rate as the original \emph{SAG}.  
Furthermore, \tool has asymptotic time complexity as efficient as gradient methods with the lowest itreation cost, and asymptotic space complexity as compact as any memory-less gradient methods.  
In practice, through extensive evaluation we demonstrated that, even without any fine-tuning or optimization of the implementation, 
\tool still outperforms both full deterministic gradient and stochastic gradient in terms of reaching the best quality optimization within the same amount of time.  
Following up, we provided evidence that full deterministic gradient and stochastic gradient would take much longer to reach a quality of optimization similar to \tool.

Currently we are extending \tool in two directions.  Both directions relate to running an iteration of full deterministic gradient in \tool.
First, we are investigating if it is beneficial to run an iteration of full deterministic gradient more often.  
In our experiments, we observed that both \emph{SG} and \emph{SAG} may converge early; the optimization may get stuck at a local sub-optimum for a long number of iterations.  
Thus we are exploring if an iteration of full deterministic gradient would get the optimization back on track in case \tool gets stuck.
Secondly, we aim to investigate how well \tool would perform in the production environment, and in distributed systems potentially running in parallel, because running full deterministic gradient even once can be prohibitive for full-scale datasets with millions to billions of non-zero entries.

% the recommender perspective 
In the future, we also aim to complete our ongoing work on the metrics perspective and on the software engineering perspective.  
Given a dataset, the quality of a recommender system is often evaluated in various metrics: 
e.g. precision, recall, area under curve, reciprocal rank, NDCG, and variants of the above such as top-K precision and top-K hit rate.
% what is NDCG?
Many papers in the literature claim their objective function is better by illustrating that their objective function performs in some of these metrics better than other objective functions.  
Therefore, in the metrics perspective, we are exploring and investigating which factors are more relevant and important towards scoring high in the various metrics: 
is it the objective function, the method for convex-optimization such as \emph{SAG}, other fine-tuning mechanisms such as bootstrapping, 
or the hyper-parameters that we use in convex-optimization.  All of these factors can be dataset-specific.  
Indeed, our inherent assumption in this paper is that a better quality optimization yields better recommender systems. 
In the future, we would like to explore if there are other factors that are more worthwhile than a fast convergence rate or a low iteration cost towards better recommender systems.  

% the software engineering perspective 
In the software engineering perspective, we study how to increase the productivity of data scientists.  
At this point, we are designing and developing a \emph{mix-n-match} or \emph{plug-n-play} framework that enables data scientists in a least effort way, 
to very rapidly prototype and experiment many different combinations of objective functions, datasets, gradient methods, hyper parameters and evaluation metrics.  


\bibliographystyle{abbrv}  
\bibliography{refs}

\end{document}
