\documentclass{sig-alternate}
\pdfpagewidth=8.5in
\pdfpageheight=11in
\usepackage[utf8]{inputenc}
\usepackage[T1]{fontenc}
\usepackage[font=bf]{caption}
\usepackage{listings}
\lstset{language=Python}
\usepackage{color}
\usepackage{xspace}
\usepackage{hyperref}
\usepackage{url,moreverb,graphicx}
\usepackage{ifthen}
\usepackage{paralist}
\usepackage{lipsum}
\usepackage{tikz}
\usetikzlibrary{patterns}
\usepackage{pgfplots}
\usepackage{comment}

\clubpenalty=10000
\widowpenalty = 10000

\definecolor{lightgray}{rgb}{.9,.9,.9}
\definecolor{darkgray}{rgb}{.4,.4,.4}
\definecolor{purple}{rgb}{0.65, 0.12, 0.82}
\lstdefinelanguage{JavaScript}{
  keywords={typeof, new, true, false, catch, function, return, null, catch, switch, var, for, if, in, while, do, else, case, break},
  keywordstyle=\color{blue}\bfseries,
  ndkeywords={class, export, boolean, throw, implements, import, this, assert, forall, exists, not, int},
  ndkeywordstyle=\color{darkgray}\bfseries,
  identifierstyle=\color{black},
  sensitive=false,
  comment=[l]{//},
  morecomment=[s]{/*}{*/},
  morecomment=[l]{\%},
  commentstyle=\color{purple}\ttfamily,
  stringstyle=\color{red}\ttfamily,
  morestring=[b]',
  morestring=[b]"
}

\lstset{
   language=JavaScript,
   backgroundcolor=\color{lightgray},
   extendedchars=true,
   basicstyle=\scriptsize\ttfamily,
   showstringspaces=false,
   showspaces=false,
   numbers=left,
   numberstyle=\scriptsize,
   numbersep=9pt,
   tabsize=2,
   breaklines=true,
   showtabs=false,
   captionpos=b,
   xleftmargin=4.0ex
}

% How about DOMColic for the tool name?
\newcommand{\tool}{\textsc{SAG-recomputed}\xspace}

\newcommand{\code}[1]{{\texttt{#1}}}
\newcommand{\js}{{Java\-Script}\xspace}\newcommand{\header}[1]{\par\smallskip\noindent\textbf{#1}}

\newboolean{showcomments}
\setboolean{showcomments}{true}
\ifthenelse{\boolean{showcomments}}
{\newcommand{\nb}[2]{
\fbox{\bfseries\sffamily\scriptsize#1}
{\sf\small$\blacktriangleright$\textit{#2}$\blacktriangleleft$}
}
}
{\newcommand{\nb}[2]{}
}
\newcommand\james[1]{\nb{James}{#1}}
\newcommand\eric[1]{\nb{Eric}{#1}}
\newcommand\ali[1]{\nb{Ali}{#1}}

\renewcommand*{\lstlistingname}{Listing}

\newcounter{head}
\newcommand{\challenge}[1]{%
  \refstepcounter{head}
  \header{\thehead.\ \ #1.}
%\newcommand{\head}[1]{\textbf{#1.}\ \ }
}


\begin{document}

% --- Author Metadata here ---
%\conferenceinfo{ISSTA 2014, Jul 21-26, 2014}{San Jose, California}
%\CopyrightYear{2014} % Allows default copyright year (20XX) to be over-ridden - IF NEED BE.
%\crdata{0-12345-67-8/90/01}  % Allows default copyright data (0-89791-88-6/97/05) to be over-ridden - IF NEED BE.
% --- End of Author Metadata ---


\title{{\ttlit ConcolicDOM:} Concolic Generation of DOM Structures for Unit Testing JavaScript}
\numberofauthors{3} 
\author{
\alignauthor
James Lo\\	
       \affaddr{Computer Science}\\
       \affaddr{University of British Columbia}\\	   
       \affaddr{Vancouver, Canada}\\
       \email{tklo@cs.ubc.ca}
\alignauthor
Eric Wohlstadter\\
       \affaddr{Computer Science}\\
       \affaddr{University of British Columbia}\\	   
       \affaddr{Vancouver, Canada}\\
       \email{wohlstad@cs.ubc.ca}
\alignauthor
Ali Mesbah\\
       \affaddr{Elec. and Comp. Engineering}\\
       \affaddr{University of British Columbia}\\	   
       \affaddr{Vancouver, Canada}\\
       \email{amesbah@ece.ubc.ca}
}
%\date{24 January 2014}

\maketitle
\begin{abstract}
%As Web applications become more prevalent in our daily lives, quality assurance of Web applications has also become more important.  
Considerable JavaScript code is written to access and update a Web application's user interface through the Document Object Model API.  
The DOM models the UI in a tree structure.  
In this paper we present our generic and browser independent approach for concolic generation of DOM trees for testing JavaScript Web applications.
Testing Web applications remains a challenge because executing different parts of JavaScript code requires different yet specifically precise DOM tree structures.
If there is any mismatch between the code and the DOM, e.g. when a DOM operation fails, entire code execution would eventually halt and the test would terminate prematurely.  
To overcome these challenges, we apply concolic techniques to generate HTML.  
We designed a DOM solver to support the 2D structure of the DOM tree, to infer implicit clues from DOM operations that are partial and incomplete, and to tailor the DOM tree for targeting precise subsets of the code base.  
We also implemented an end to end automatic system from deducing constraints to generating HTML and driving test execution because the number of unique DOM trees can grow as exponentially as the number of execution paths.
We conducted a case study on the DOMtris application in which we will show how our approach significantly improves path coverage that includes a part of JavaScript code that implements a core functionality of the application.
\end{abstract}

% A category with only the three required fields
\category{D.2.5}{Software Engineering}{Testing and Debugging}%[Symbolic execution, Test coverage of code, Test execution]
%A category including the fourth, optional field follows...
%\category{D.3.2}{Software}{Programming Languages}[JavaScript]

\terms{Design, Algorithms, Experimentation}

\keywords{Testing, concolic execution, DOM, JavaScript}

\section{Introduction}
% Why RecSys
Shopping, text or display advertising, renting movies, listening to music… recommender systems are prevalent and ubiquitous in our daily lives.  
% Why Matrix Factorization  
Matrix factorization (\emph{MF}) is a popular technique in model-based recommender systems.  
Indeed, MF has been utilized extensively in past research for handling both explicit \cite{mmmf2005fast, mnar, gapfm} ratings, 
and implicit \cite{wrmf2008hu, wrmf2008pan, climf, bpr, mnar} feedback.  

% Pitfalls with FG and SG  
In recommender systems that utilize matrix factorization, most optimize an objective function.  
Full deterministic gradient \emph{FG} and stochastic gradient \emph{SG} are the two main gradient methods for optimization.  
All of the recommender systems that we mentioned utilize one of \emph{FG} or \emph{SG}.  
However, both full deterministic gradient and stochastic gradient have pitfalls when it comes to prototyping recommender systems.  

Full deterministic gradient can offer high quality optimizations.  
However, FG is slow because in each iteration of optimization, FG has to pass through all the samples in the dataset.  
Stochastic gradient is relatively fast; iteration cost is low because each iteration of SG looks at only one or a few samples.  
However, the trade-off with SG is that it often provides relatively low quality optimizations.  

High quality optimizations with a low iteration cost is important when building recommender systems.  
The first reason is that data scientists often have to run repeated experiments: e.g. with different objective functions, different metrics, different datasets, and different optimization parameters.  
The second reason is that product life cycles are shortening in the age of agile software engineering.  
Thus data scientists are often faced with the challenge of running more experiments and producing high quality results with less time.  

%Why SAG in prototyping model-based recommender systems  
In this paper, we study the challenge from the perspective of convex-optimization.  
We propose using the stochastic average gradient \emph{SAG} method \cite{schmidt2013minimizing, roux2012stochastic} as a viable alternative to using \emph{FG} and \emph{SG} during the prototyping process.  
\emph{SAG} has the distinctive advantage that its optimization quality is proven to be much better than \emph{SG}; at the same time \emph{SAG}'s iteration cost is asymptotically identical to \emph{SG}.  
However, adapting \emph{SAG} to matrix factorization is difficult because \emph{SAG} requires previously-computed gradients and storing these gradients can lead to very high asymptotically space complexity.  
We explore the challenge with space-complexity, and addresses it by proposing a re-computation approach \tool that re-computes the previously-computed gradients.  
Our re-computation approach preserves the low iteration cost of \emph{SAG}.  
The asymptotic space complexity of our \tool approach is as efficient as the memory-less full deterministic gradient, and stochastic gradient.  

To the best of our knowledge, we are the first to
\begin{compactitem}
\item Identify the pitfalls associated with using full deterministic gradient and stochastic gradient when data-scientists prototype model-based recommender systems.  
\item Propose Stochastic Average Gradient (\emph{SAG}) as a viable alternative for yielding faster convergence in the convex-optimization of matrix factorization.  
\item Extend SAG into \tool for matrix factorization, resolve the space complexity challenge in adapting SAG from the domain of large-scale supervised-machine-learning into the domain of prototyping recommender objective-functions.  
\item Prove in theory, that our \tool approach has identical asymptotic time complexity and identical asymptotic space complexity as stochastic gradient.  
\item Extensively evaluate and compare SAG-MF across multiple RecSys objective functions and diverse datasets.  
\item Demonstrate in practice that, even before any optimization of the implementation, our SAG-MF algorithm still yields faster convergence despite the additional time of re-computation, and that SAG-MF uses memory at a level similar to stochastic gradient.  
\end {compactitem}

\section{Challenge}
When applying \emph{SAG} into matrix factorization, a main challenge is to make available the fine-grain gradients of previously-sampled entries: 
$\bar{m}_{entry(b).i}^{t}$ from equation \ref{eq:sag_mu}, and 
$\bar{m}_{entry(b).j}^{t}$ from equation \ref{eq:sag_mv}


\section{approach}
Similar to the chain-rule approach, \tool re-computes $\bar{m}_{entry(b).i}^{t}$ in equation \ref{eq:sag_mu} and $\bar{m}_{entry(b).j}^{t}$ in equation \ref{eq:sag_mv}:

\begin{equation} \label{eq:sagre_mu}
  \bar{m}_{U}^{t+1} = \bar{m}_{U}^{t} + \sum_{b=1}^{B}\left[\bar{m}_{entry(b).i}^{t+1} - \frac{\text{d}f(\bar{u}_{entry(b).i}^{s}, \bar{v}_{entry(b).j}^{s})}{\text{d}\bar{u}_{entry(b).i}^{s}} \right]
\end{equation}

The main difference is that, all 

The chain-rule approach must store $min(M,N)$ different copies of past versions of $\bar{m}_{entry(b).i}^{t}$.
There are two reasons.  First, each entry can come from a different iteration.  Second, the same entry may get sampled at more than one different iterations.  
\tool resolves this issue, by predicting ahead the entires that we are going to sample.  


\emph{Theorem 5.}
\tool has convergence rate at least as fast as \emph{SAG}.
\begin{proof}

\end{proof}


\emph{Theorem 6.}
\tool has $\theta(1)$ asymptotic time-complexity and is as efficient as both \emph{SAG} and stochastic gradient.
\begin{proof}
\tool achieves the lowest possible asymptotic iteration cost.
\end{proof}


\emph{Theorem 7.}
\tool has $\theta(N + nDims*(nRows+nCols))$ asymptotic space-complexity and is as compact as any memory-less gradient method.
\begin{proof}
Indeed, \tool achieves the best possible asymptotic space-complexity (\emph{Theorem 2}).
\end{proof}

\section{DOM Solver}
The DOM solver takes the constraints defined by the MapDeducer and attempts to generate a satisfiable DOM structure.  The solver is implemented as an extension of a SMT solver~\cite{cvc3} and would report anything not satisfiable.  

\header{DOM Tree \& DOM Operations.}
Recall a major part of the DOM is its single parent, multi-children tree structure.  When generating a satisfiable DOM, we use the execution of DOM operations to infer the overall DOM tree.    
Each DOM operation in any line of code is like a piece of a puzzle describing a subset clue of the overall DOM tree.   
For example {\tt a = elem.parentElement.nextElementSibling} implies 2 subset clues: {\tt elem} has a parent element, and the parent has a sibling.  
Note that when the condition is {\tt a = elem.parentElement.nextElementSibling === null}, then the clues become {\tt elem} has a parent element, yet the parent has no next sibling and thus is the last child.

That said, questions remain unanswered about exactly where does {\tt elem} fit in or belong in the overall DOM tree; and other DOM operations would provide clues for that.  
The DOM solver would take all the clues and generate a satisfiable tree structure.   

\begin{figure}
\begin{lstlisting}[caption=HTML generated for guiding the execution to follow the {\tt true} branch in the {\tt if} statement in Sample Code\ref{dom0}.label=htmlExtended]  
<span id="c">
  <span/>
  <span id="b">
    <span id="d">
      <span id="elem"/>
    </span>
  </span>
  <span/>
</span>
\end{lstlisting}
\end{figure}


% XPath
% how to create rules in DOM solver, example rules.  Intuition behind it.
% why not XML solver, not scalable
% SMT-lib language, swappable between CVC and Z3.
	% take advantage of Moore's law in terms of hardware performance and breakthroughs in constraint solvers
	% future compatibility for multiple data types

\header{DOM Operations into SMT Quantifiers.}	
In the solver we transform each DOM operation into a SMT function.  We then use quantifiers (e.g. {\tt EXISTS}, {\tt FORALL}) to define how the SMT functions relate to each other.  
Sample Code ~\ref{childrenLength} shows the boolean functions and integer functions we defined for supporting the {\tt elem.children.length} operation.  We first quantify the parent-child relationship: 
\begin{compactitem}
\item a node cannot be a child of itself, see {\tt line 1} and {\tt line 4} in Sample Code ~\ref{childrenLength}.
\item a child of a node cannot be the node's parent at the same time: {\tt line 8}.
\item a child can have only 1 parent: {\tt line 13}.
\end{compactitem}
Next, we define how children are ordered and quantify {\tt children.length}:
\begin{compactitem} 
\item first child starts at position or index 0: {\tt line 18} and {\tt line 24}.
\item last child has the largest child index: {\tt line 30}.
\item {\tt children.length} equals to one plus the child index of the last child, because the first child starts at position 0: {\tt line 40}.
\end{compactitem}
The parent SMT functions are quantified as the inverse of the child SMT functions (e.g. {\tt line 48}).  Similar to {\tt firstChild()} and {\tt lastChild()}, the sibling SMT functions are defined by extending {\tt children(x, y, j)}.  
For example, the next sibling of a node has the same parent and child index {\tt j+1}, when the node has child index {\tt j}.

\begin{figure}[th]
\begin{lstlisting}[caption=SMT functions for defining the children.length DOM operation.  We start with defining the parent-child relationships; then move on to the ordering of children; then use the child index of the last child to define and quantify the {\tt childrenLength()} boolean function,label=childrenLength.]  
% child(x, y): x is a child of y.
child: (Node, Node) -> BOOLEAN;	

% x cannot be a child of itself.
ASSERT FORALL (x: Node):	
  NOT(child(x,x));
	
% when y is the parent of x,
% then y cannot be a child of x.  
ASSERT FORALL (x, y: Node):
  child(x,y) => NOT(child(y,x));
  
% a child has only 1 parent.
ASSERT FORALL (x, y, z: Node):
  (child(x,y) AND DISTINCT(y,z)) 
    => NOT(child(x,z));

% x is the j-th child of y.
children: (Node, Node, INT) -> BOOLEAN;
ASSERT FORALL (x,y:Node, j:INT): 
  children(x, y, j) => 
    child(x, y) AND j >= 0;

% child position/index starts at 0.
firstChild:		(Node, Node) -> BOOLEAN;
ASSERT FORALL (x, y:Node):
  firstChild(x, y) <=> 
    children(x, y, 0);

% every other child must have an index 
% smaller than that of the last child.
lastChild: (Node, Node) -> BOOLEAN;	
ASSERT FORALL (x, y:Node): 	
	lastChild(x, y) => EXISTS(j:INT): 
	  children(x, y, j) AND 
	  (FORCALL(z:Node, k:INT): 
	    (children(z, y, k) AND 
	    DISTINCT(z, x)) => k < j);

% children.length equals to 1 plus
% the child index of the last child.
childrenLength:	(Node) -> INT;
ASSERT FORALL (y:Node, j:INT):
	childrenLength(y) = j <=> 
	  EXISTS(x:Node): (lastChild(x, y) 
	    AND children(x, y, j-1));

% example of inversion
% y is the parent of x, is the same as 
% x is a child of y.
parent: (Node, Node) -> BOOLEAN; 
ASSERT FORALL (x, y: Node): 
  parent(y, x) <=> child(y, x); 
\end{lstlisting} 
\end{figure}

\header{Negation.}
Negations are supported by SMT solvers by default.  To negate a constraint we wrap the constraint with {\tt NOT()}.
%While it is possible to define a customized data structure to mimic the DOM tree in CVC, the boolean/integer function approach is simple and convenient for negating conditions and going into different execution paths.
%Because Web browsers also support accessing the DOM tree via XPath~\cite{documentEvaluate}, we defined additional SMT functions: {\tt following\_sibling}, {\tt preceding\_sibling}, {\tt descendant} and {\tt ancestor}.

\header{Solver Output into HTML.}
Because we used Boolean functions and Integer functions, CVC is only going to solve for the interdependent logic constraints and it does not directly yield a concrete DOM tree.  
Instead, CVC only expands the quantifiers and it outputs more {\tt ASSERT} statements for making the constraints satisfiable.  

In Sample Code \ref{domOr}, the condition inside the first {\tt if} statement has 2 sub conditions: 
\begin{compactitem}
\item {\tt d === elem.firstElementChild} ({\tt line 3}), and 
\item {\tt d === b.lastElementChild} ({\tt line 4}).
\end{compactitem}
When the DOM solver solves this {\tt if} statement for going to the {\tt true} branch, it would output {\tt "ASSERT lastChild(d, b);"} because it has decided on making the second sub condition ({\tt line 4}) {\tt true}.

To build the DOM tree, we built an API that parses the CVC output text and builds a model for the DOM.  
In the CVC output, CVC creates many temporary variable names, thus each DOM element can have more than one alias.  
To consolidate the aliases, we start with DOM operations expressing the parent child relationship because each child can have only 1 parent.  
We also used other deterministic DOM operations such as {\tt firstChild()} and {\tt lastChild()} to group aliases together.  
For example, if {\tt x} is the first child of {\tt y} and {\tt z} is also the first child of {\tt y}, then {\tt x} and {\tt z} are two aliases of the same DOM element.  

Once the parent child relationship is established, our next step is to organize the ordering of children.  
Some DOM children have their positions explicit (e.g. {\tt firstChild(x, y)}, {\tt lastChild(x, y)} and {\tt children(x, y, j)}), yet very often the child parent relationship is implied.  For example, {\tt nextSibling(x, z)} implies {\tt x} and {\tt z} share the same parent.  
To calculate the ordering of children, we use the explicitly positioned children as anchors and we relate the anchors to other children by the sibling operations.  

%In case of any uncertainty, we would query the CVC model.  We have to do a binary search because querying the CVC model supports only {\tt true} or {\tt false} answers.  
%Because CVC does not support strings natively, for now we map tag names into integers in CVC.  <span> is the default tag type because <span> is an inline element and not a block.
%Once the DOM tree is defined, we search for the root of the DOM tree (the element without a parent) and generate HTML.  

% \header{DOM mutations.}

% \header{Conditional Slicing.}

\section{Implementation}
\label{impl}
\header{Concolic Driver.}
The concolic driver automates and coordinates the concolic process end to end from deducing constraints to generating HTML and driving execution.  Specifically, the concolic driver would repeatedly
\begin {compactitem}
\item Open the target URL in the Web browser
\item Load the generated html (initially empty html)
\item Execute the target JavaScript code
\item Measure coverage, and decide which path to go next
\item Call the MapDeducer, which returns the DOM constraints in text
\item Send the constraints to the DOM solver, which returns a satisfiable DOM tree
\item Go back to the first step with the newly generated HTML.
\end{compactitem}
The concolic driver would iterate the cycle until it has reached a certain point, which can be configured to be a specific number of iterations (e.g. 1,000 DOM trees generated), a certain level of coverage (e.g. 100\% branch coverage), or both. 


\header{Instrumenting and Executing JavaScript.}  
Our approach has to be generic, transparent and browser-independent.  
We use Selenium's WebDriver \cite{webdriverjs} to drive Web browsers for executing JavaScript.  
WebDriver runs on multiple browsers, thus using WebDriver meets our design goal of being browser-independent.  

When JavaScript is getting downloaded onto a Web browser, we use the WebScarab proxy \cite{webscarab} to intercept the download and to instrument code.  
The proxy passes intercepted JavaScript code to the Google Closure Compiler API \cite{ClosureCompiler}, which transforms JavaScript into calls to the operator functions. 

Both the Backward Slicer and MapDeducer are implemented as JavaScript APIs, and WebDriver natively supports calling JavaScript functions within the browser.  
Thus our approach is entirely transparent and can be applied to multiple brands of Web browsers.    


\header{Inline JavaScript \& {\tt eval()}.}  
In addition to source files, JavaScript code can also be found within {\tt eval()} and inlined as attributes of a DOM element inside the HTML declaration (e.g. {\tt <body onload="runFunction()">}).
We instrument each {\tt eval(code)} statement into {\tt eval(instrument(code))}.  We do not override the native {\tt eval()}, because the native {\tt eval()} must be called inside the closure to give {\tt code} access to the closure's local variables.  
The {\tt instrument()} function would send the {\tt code} to the proxy for instrumentation via a XMLHttpRequest.  

To instrument inline JavaScript, we traverse the webpage's original existing HTML using the JSoup API~\cite{jsoup}.  
Once we detect DOM attributes that contain JavaScript, we pass the JavaScript to the Google Closure Compiler API for instrumentation.  
For newly created elements, e.g. {\tt elem.innerHTML = text}, we use getters and setters to detect the new elements.  
Once detected, we traverse the new elements, extract JavaScript and call the {\tt instrument()} function.  
% Tudu


\header{DOM Solver.}  
CVC allows writing the constraints in Java, yet we do not want to hardcode the constraints because the constraints are different for each execution path.  
Thus we use Java's ProcessBuilder~\cite{processbuilder} class to communicate with the executable (.exe) version of CVC.  We decided to use CVC3 \cite{cvc3} rather than CVC4~\cite{cvc4} because CVC3 is generally more stable during our experimentation.  
The API that parses the CVC output is also implemented in Java, thus we used the W3C DOM API~\cite{DomAPI} for building a satisfiable DOM tree and generating HTML.  
%QUnit provides a <div> called the fixture which can be set as the HTML for a test case.   Before execution, WebDriver would set the {\tt innerHTML} of the fixture element to the generated HTML.  


\header{Limited Path Coverage.}  
% zero, 1 and n.
Very often we would not know how many times to execute a loop.  For example, in Sample Code ~\ref{dom0}, there is no upper limit to the number of children that {\tt field} can have.  
\tool would execute loops zero, one, and {\tt n} times, where {\tt n} can be configured for a particular loop or for all loops.  Thus \tool would achieve limited path coverage rather than full path coverage.  


%\header{Indexing Functions.}  
%Most of the time JavaScript functions are defined inside closures and are not accessible~\cite{privatefunctions}.  
%When we instrument JavaScript code, we extract the functions by assigning them to an object that we have access to.  
%Because functions can have the same name, we use the node number in the JavaScript Abstract Syntax Tree as ID.

%\header{QUnit}
%\tool has integration with QUnit~\cite{qunit} so that existing test suites can automatically take advantage of \tool without additional manual effort.  
% \tool is also extensible to be integrated with other test frameworks~\cite{jstests}.
% Thus given a test case, be its inputs were generated manually or automatically, \tool can be used to help the test case and its assertions get fully utilized.

\header{Inserting HTML into Webpage.}
In HTML, certain DOM elements must be wrapped inside other DOM elements~\cite{jsninja}.  

For example, table cells ({\tt <td>} and {\tt <th>}) have to be contained inside a table row ({\tt <tr><td></td></tr>}).
Table rows ({\tt <tr>}) must be contained inside 
\begin{compactitem}
\item a table header: {\tt <thead> <tr></tr>... </thead>}, 
\item a table body: {\tt <tbody> <tr></tr>... </tbody>}, or
\item a table footer: {\tt <tfoot> <tr></tr>... </tfoot>}.
\end{compactitem}
\sloppy{
Similarly, every table header, table body, and table footer must be contained inside a table too: 
{\tt <table>} {\tt <thead>...</thead>} {\tt <tbody>...</tbody>} {\tt <tfoot>...</tfoot>} {\tt </table>}.
Each table contains at most 1 header, 1 body and 1 footer.  
}

{\tt <option>} and {\tt <optgroup>} have to be inside a {\tt <select multiple="multiple"> ... </select>}.
{\tt <legend>} has to be inside a {\tt <fieldset> ... </fieldset>}.
\tool processes the generated HTML before inserting it into a webpage.  

\section{Evaluation}


\subsection{Research Questions}

\header{Quality vs. Time.}
Our first set of research questions focuses on optimization quality vs. time.
Given more time, any gradient method yields a better optimization.  
Our focus here is to identify which gradient method is the most suitable for data-scientists prototyping recommender systems.
In terms of suitability, we mean the gradient method that yields the best quality optimization within the shortest amount of time.
Here, we consider the general \emph{SAG} approach \emph{as is}. 
The next set of research questions studies the specific \emph{space vs. time} trade-off between \tool and the na$\ddot{i}ve$ approach to \emph{SAG}.

Between \emph{SAG}, full deterministic gradient and stochastic gradient,
\begin{sloppy}
\begin{compactenum}
\item Which gradient method yields a better optimization given the same amount of time?
\item Which gradient method uses the shortest amount of time to reach a similar quality of optimization?
\item Can \emph{SAG} and specifically \tool work well with different objective functions in recommender systems?
\item Can \emph{SAG} and specifically \tool work well with different matrix datasets?
\end{compactenum}
\end{sloppy}


\header{Space vs. Time.}
Our second set of research questions investigates whether re-computing is worth the additional time.
Here, we investigate the actual space vs. time trade-off between \tool vs. the na$\ddot{i}$ve approach to \emph{SAG}:

Compared to the na$\ddot{i}ve$ approach to SAG, in practice
\begin{sloppy}
\begin{compactenum}
\setcounter{enumi}{4}
\item How much slower is \tool due to re-computing?
\item How much memory does \tool save?
\end{compactenum}
\end{sloppy}



\subsection{Experimental Setup}

\header{Distinct Objective Functions.}
The objective functions we choose already uses full deterministic gradient (\emph{FG}) or stochastic gradient (\emph{SG}).  
In general, any function that is differentiable, and specifically any function that uses (\emph{FG}) or (\emph{SG}) can use \emph{SAG} and \tool.
If a function is convex, then gradient methods guarantee a global optimum over time.
The functions we have chosen are distinct from each other.  The goal is to illustrate \emph{SAG} and \tool are capable of working with different objective functions.
% cite objective functions
\begin{sloppy}
\begin{compactenum}
\item \emph{Least-squares}: L2 and its variants \cite{mnar, wrmf2008hu, wrmf2008pan} are popular objective functions when building recommender systems.
\item \emph{CLiMF} \cite{climf}: Collaborative-Less-is-More-Filtering uses ordinal logistic regression to smooth the mean reciprocal rank function and to learn how a user ranks different items; 
CLiMF performs gradient ascent because the optimization goal is to maximize an objective function.
\item \emph{BPR-MF} \cite{bpr}: Bayseian Personalized Learning has an objective function that minimizes 
the difference between any two \emph{item} ratings (column entries) of the same user (same row).
BPR-MF performs gradient descent.
\end{compactenum}
\end{sloppy}


\header{Diverse Datasets.}
Our datasets are binary data that serve as implicit feedback in recommender systems. 
They represent diverse relationships including trustees \cite{epinions}, webpage bookmarking \cite{digg12month1}, casting \cite{IMDB}, social network \cite{ljournal2008}, and linking webpages \cite{wikipedia20070206}.
The datasets come from the Sparse Matrix collection at the University of Florida.
% cite datasets
\begin{sloppy}
\begin{compactenum}
\item \emph{Epinions} \cite{epinions}: $A(i,j) = 1$ when user $i$ is a trustee of user $j$, $A(i,j) = 0$ otherwise.  
The trustee relationship is not necesseary mutual.  The epinions dataset is identical to the epinions dataset that Shi et al. used in \cite{climf}. 
\item \emph{Digg12month1} \cite{digg12month1}: $A(i,j) = 1$ when user $i$ tags webpage $j$ as favorable; 0 represents no opinion. 
\item \emph{IMDB} \cite{IMDB}: $A(i,j) = 1 $ if movie $i$ has actor or actress $j$ as cast, $A(i,j) = 0$ otherwise. 
\item \emph{Live Journal} \cite{ljournal2008}:  $A(i,j) = 1 $ if user $i$ has user $j$ as his friend, $A(i,j) = 0$ otherwise. 
The graph is directed because the friendship is not neceseary mutual.
\item \emph{Wikipedia} \cite{wikipedia20070206}: $A(i,j) = 1$ if page $i$ links to page $j$, $A(i,j) = 0$ otherwise.  
\end{compactenum}
\end{sloppy}


\header{Hyper Parameters.}
For the purpose of comparison, we standardize all hyper-parameters across all objective functions, all datasets, and all gradient methods.
The only exception is that we run full deterministic gradient \emph{FG}) for only 500 iterations vs. 5,000 for stochastic gradient (\emph{SG}) and \emph{SAG}.

Convergence theory guarantees that given the same number of iterations, \emph{FG} yields a much better quality optimization than \emph{SG}.
However, our goal is to identify the gradient method that yields the best quality optimization within the shortest amount of time.
Therefore, we want to see whether \emph{FG} would take longer to yield a similar quality of optimization as \emph{SG}, and how much longer.
Through experience with our objective functions and datasets, we observed that 500 iterations \emph{FG} yields a similar quality of optimization as \emph{SG}.
As a result, we run \emph{FG} to 500 iterations, and compare how much longer 500 iterations of \emph{FG} would take than 5000 iterations of \emph{SG}.

\begin{sloppy}
\begin{compactitem}
\item Step size or learning rate: 0.0001
\item Regularization $\lambda$: 0.001; $\lambda$ is identical for regularizing both \emph{user} matrix $U$ and \emph{item} matrix $V$
\item Iterations: 5000 for \emph{SG} and \emph{SAG}, which is roughly 10\% of the number of non-zero entries in each sub-dataset.
\item Latent dimensions ($nDims$): 5 
\end{compactitem}
\end{sloppy}

For gradient descent, step size is $\alpha < 0$ and $\lambda > 0$;
for ascent, step size is $\alpha > 0$ and $\lambda < 0$.


\header{Hardware and OS.}
A MacBook Pro run all experiments that study optimization \emph{Quality vs. Time}.
Our MacBook Pro is the Late 2013 15-inch version \cite{macbookprolo}; it has OS-X Yosemite, 2.3Ghz Intel i7 quad-core CPU, 16GB RAM, and a Nvidia 750M GPU.

When studying \emph{Space vs. Time}, we measure memory usage after the first iteration.  
Initially we plan to run all experiments on the MacBook Pro.
However, the memory-profiing feature of Matlab works only on Windows.  

For the sub-datasets, we run the \emph{memory} experiment on a Dell XPS 12 \cite{dellxps12} laptop.
The Dell XPS 12 has Windows 8.1, 1.6Ghz Intel i5 dual-core CPU, 4GB RAM, and integrated graphics.  

For the full datasets, we run the \emph{memory} experiment on a remote server that has more RAM.
Our remote server has Windows Server 2008R2, 2.50Ghz Intel Xeon 2x quad-core CPUs (total 8 CPU cores) and 16GB RAM.

Both MacBook Pro and remote server have Matlab R2014a; Dell XPS 12 has Matlab R2012a.
All 3 computers have the Matlab parallel computing toolkit.



\subsection{Quality vs. Time}
%\header{Methodology.}
For the purpose of comparison, we fix the seed for generating random numbers so that 
Measure optimization quaity in each iteration. report the best optimixatiom 


%\section{Future Work}  

\header {Solving DOM Attributes.}
Most DOM Attributes take the form of integers and strings.  

\header{Element vs. Node.}
% inheritance in CVC

\header{DOM Mutations.}
% conditional slicing

\header{Multiple Data Types.}

\header{Designer-Developer Collaboration.}
The majority of JavaScript bugs are DOM related~\cite{frolin2013}.
Ultimately, our higher level goal is to foster closer collaboration among designers and developers.\footnote{As part of separating concerns, design and development are often done by distinct individuals having very different backgrounds.}
For example, because \tool generates reference HTML for satisfying code execution, it can be used to detect DOM mismatches between the designer's HTML vs. what is expected in the developer's JavaScript code.  
A mismatch may not always be the developer's fault.  Sometimes it is possible that the designer may have made a mistake when updating the HTML or may be just too busy and have forgotten to notify the developer about a change.  

%XML, other programming languages 

% how about HTML5 canvas

\header{Other SMT Solvers.}
Microsoft Z3~\cite{z3}.  SMT-lib language.







\section{Related Work}


\section{Future Work \& Conclusion}
This paper is the first in the series of our study on data scientists prototyping model-based recommender systems.  
We explored the convex-optimization perspective of the problem: we propose Stochastic Average Gradient as a viable alternative to Full Deterministic gradient and Stochastic gradient.  
By taking advantage of \emph{SAG}'s fast convergence rate and low iteration cost, we aim to enable data-scientists run more experiments and produce high quality results with less time.  
In theory, we proved that our extension and adaptation of \emph{SAG} preserves the fast convergence rate as the original \emph{SAG}.  
Furthermore, \tool has asymptotic time complexity as efficient as gradient methods with the lowest itreation cost, and asymptotic space complexity as compact as any memory-less gradient methods.  
In practice, through extensive evaluation we demonstrated that, even without any fine-tuning or optimization of the implementation, 
\tool still outperforms both full deterministic gradient and stochastic gradient in terms of reaching the best quality optimization within the same amount of time.  
Following up, we provided evidence that full deterministic gradient and stochastic gradient would take much longer to reach a quality of optimization similar to \tool.

Currently we are extending \tool in two directions.  Both directions relate to running an iteration of full deterministic gradient in \tool.
First, we are investigating if it is beneficial to run an iteration of full deterministic gradient more often.  
In our experiments, we observed that both \emph{SG} and \emph{SAG} may converge early; the optimization may get stuck at a local sub-optimum for a long number of iterations.  
Thus we are exploring if an iteration of full deterministic gradient would get the optimization back on track in case \tool gets stuck.
Secondly, we aim to investigate how well \tool would perform in the production environment, and in distributed systems potentially running in parallel, because running full deterministic gradient even once can be prohibitive for full-scale datasets with millions to billions of non-zero entries.

% the recommender perspective 
In the future, we also aim to complete our ongoing work on the metrics perspective and on the software engineering perspective.  
Given a dataset, the quality of a recommender system is often evaluated in various metrics: 
e.g. precision, recall, area under curve, reciprocal rank, NDCG, and variants of the above such as top-K precision and top-K hit rate.
% what is NDCG?
Many papers in the literature claim their objective function is better by illustrating that their objective function performs in some of these metrics better than other objective functions.  
Therefore, in the metrics perspective, we are exploring and investigating which factors are more relevant and important towards scoring high in the various metrics: 
is it the objective function, the method for convex-optimization such as \emph{SAG}, other fine-tuning mechanisms such as bootstrapping, 
or the hyper-parameters that we use in convex-optimization.  All of these factors can be dataset-specific.  
Indeed, our inherent assumption in this paper is that a better quality optimization yields better recommender systems. 
In the future, we would like to explore if there are other factors that are more worthwhile than a fast convergence rate or a low iteration cost towards better recommender systems.  

% the software engineering perspective 
In the software engineering perspective, we study how to increase the productivity of data scientists.  
At this point, we are designing and developing a \emph{mix-n-match} or \emph{plug-n-play} framework that enables data scientists in a least effort way, 
to very rapidly prototype and experiment many different combinations of objective functions, datasets, gradient methods, hyper parameters and evaluation metrics.  


\bibliographystyle{abbrv}  
\bibliography{refs}

\end{document}
