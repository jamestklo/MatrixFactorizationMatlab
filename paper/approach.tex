\section{approach}
Similar to the chain-rule approach, \tool does not store and re-computes $\bar{m}_{entry(b).i}^{t}$ in equation \ref{eq:sag_mu} and $\bar{m}_{entry(b).j}^{t}$ in equation \ref{eq:sag_mv}:

\begin{equation} \label{eq:sagre_mi}
  \bar{m}_{entry(b).i}^{t} =\:recomputed\:\:\frac{\text{d}f(\bar{u}_{entry(b).i}^{s}, \bar{v}_{entry(b).j}^{s})}{\text{d}\bar{u}_{entry(b).i}^{s}}
\end{equation}

\begin{equation} \label{eq:sagre_mj}
  \bar{m}_{entry(b).j}^{t} =\:recomputed\:\:\frac{\text{d}f(\bar{u}_{entry(b).i}^{s}, \bar{v}_{entry(b).j}^{s})}{\text{d}\bar{u}_{entry(b).j}^{s}}
\end{equation}

The chain-rule approach is undesirable because it must store $min(M,N)$ different copies of past versions of $\bar{m}_{entry(b).i}^{t}$ and $\bar{m}_{entry(b).j}^{t}$.
There are two problems.  First, each entry can come from a different iteration; or different entries can come from different iterations.
Second, the same entry may get sampled more than once at two or more different iterations.

To save space, we must store as few copies of $\bar{m}_{entry(b).i}^{t}$ and $\bar{m}_{entry(b).j}^{t}$ as possible.
\tool resolves the two problems above with two steps.
First, \tool predicts ahead the entires that we are going to sample.
Second, \tool performs a full deterministic gradient \emph{FG} just before \tool re-samples the same entry.

At the iteration that \tool performs a full deterministic gradient, we call it iteration $s$, \tool stores 4 matrices:
\begin{sloppy}
\begin{compactitem}
\item the actual \emph{user} matrix $U$ at iteration $s$: $U^{s}$
\item the actual \emph{item} matrix $V$ at iteration $s$: $V^{s}$
\item aggregated memory gradient for \emph{user} matrix $U$: $\bar{m}_{U}^{s}$
\item aggregated memory gradient for \emph{item} matrix $V$: $\bar{m}_{V}^{s}$
\end{compactitem}
\end{sloppy}

We should distinguish that $U^{s}$ and $V^{s}$ are stored \emph{just before} \tool performs a full deterministic gradient at iteration $s$.
The significance is that we will use $U^{s}$ and $V^{s}$ to re-compute the fine-grain memory gradients at future iterations $t > s$.

$\bar{m}_{U}^{s}$ and $\bar{m}_{V}^{s}$ are the direct outcome results of the full deterministic gradient.  
The reason is that \emph{FG} samples all $N$ entries and thus resets every possible fine-grain gradient in memory.
Thus we store $\bar{m}_{U}^{s}$ and $\bar{m}_{V}^{s}$ \emph{after} \tool performs an iteration of \emph{FG}.

At the iterations $t$ in between \tool performs two \emph{FG}'s, e.g. $s < t < s'$, \tool performs iterations of ordinary \emph{SAG}.
When \tool performs ordinary \emph{SAG}, \tool computes but does \textbf{not store} the latest version of fine-grain gradients of individual entries: 
\[
\bar{m}_{entry(b).i}^{t+1} = \frac{\text{d}f(\bar{u}_{entry(b).i}, \bar{v}_{entry(b).j})}{\text{d}\bar{u}_{entry(b).i}}\:in\:equation\:\ref{eq:sag_mi}
\]
\[
\bar{m}_{entry(b).j}^{t+1} = \frac{\text{d}f(\bar{u}_{entry(b).i}, \bar{v}_{entry(b).j})}{\text{d}\bar{v}_{entry(b).j}}\:in\:equation\:\ref{eq:sag_mj}
\]
\tool simply updates $\bar{m}_{U}^{s}$ and $\bar{m}_{V}^{s}$ with the newly computed fine-grain gradients, as equation \ref{eq:sag_mu} and equation \ref{eq:sag_mv} show.

After we perform an iteration of \emph{FG}, we predict upcoming entries ahead of time.
Therefore, at future iterations $t > s$ after a \emph{FG}, we ensure that the different entries that we are going to sample are \textbf{distinct} before we perform another iteration of full deterministic gradient.
The significance of having \emph{distinct} entries is that, at future iterations $t > s$, we will not overwrite any fine-grain gradient of individual entries:
e.g. $\bar{m}_{entry(b).i}^{t}$ in equation \ref{eq:sag_mu} and $\bar{m}_{entry(b).j}^{t}$ in equation \ref{eq:sag_mv}.
Therefore, we can \emph{re-compute} all possible fine-grain gradients of individual entries 
from a single copy of the \emph{user} matrix $U^{s}$ and the \emph{item} matrix $V^{s}$, 
that \tool stored at the same iteration $s$.

Before we perform another iteration of \emph{FG}, we do not store any fine-grain gradient $\bar{m}_{entry(b).i}^{t+1}$ or any $\bar{m}_{entry(b).j}^{t+1}$.
The reason is that we do not ever need them: \tool ensures that we will perform an iteration of \emph{FG} before we re-sample any identical entry. 
The purpose of an iteration of \emph{FG} at iteration $s' > t$ is to reset all fine-grain gradients of individual entries at the same iteration $s'$.
This way we will not need any of the fine-grain gradients at iterations $t < s'$ because we will not visit the same entries again until after we do a full reset.
Not storing the newly-computed fine-grain gradients saves $\theta(min(M,N)*nDims)$ space.

\tool re-computes the indivdiaul fine-grain gradients from the raw $U^{s}$ and $V^{s}$ matrices; doing so preserves generality.
We do not use the chain-rule: not all objective functions is compatible with it.
Both \cite{climf, bpr} do not work with the chain-rule because computing the fine-grain gradient of an entry requires not just ($\hat{a}_{ij} = \bar{u}_{i}*\bar{v}_{j}$), 
but also ($\hat{a}_{ik} = \bar{u}_{i}*\bar{v}_{k}$) for all $k \neq j$.

Next we prove \tool preserves the theoretical advantages of \emph{SAG}, and \tool is compact in space.

\emph{Theorem 5.}
\tool has convergence rate at least as fast as \emph{SAG}.
\begin{proof}
The proofs of \emph{SAG}'s convergence rates \cite{schmidt2013minimizing, roux2012stochastic} do not restrict where the starting points are for optimization.
In matrix factorization, the meaning is that we can start \emph{SAG} with any (random) matrices $U$ and $V$ (e.g. $U^{S}$ and $V^{S}$)
and still experience the convergence rates of \emph{SAG}.
Therefore, at iterations that \tool performs \emph{SAG}, \tool has convergence rate equal to \emph{SAG}.
Similarily, the convergences rates of full determinsitic gradient (\emph{FG}) allows any $U$ and $V$ as the starting matrices.
Therefore, when \tool performs \emph{FG}, \tool inherits the convergence rates of \emph{FG}.
\emph{FG} has the fastest convergence rates.
Therefore, at any iteration, \tool has convergence rates at least as fast as \emph{SAG}.
\end{proof}


\emph{Theorem 6.}
\tool has $\theta(1)$ time-complexity and is asymptotically as efficient as both \emph{SAG} and stochastic gradient.
\begin{proof}
At iteartions that \tool performs \emph{SAG}, we totally re-compute the past versions of the fine-grain gradients for the same batch of samples. 
The re-computing done by \tool essentially doubles the amount of computation compared to \emph{SAG} and stochastic gradient.
Doubling the amount of computation multiplies time-complexity by only a constant; 
thus \tool preserves the low iteration cost of \emph{SAG}.

The interesting case is when \tool performs an iteration of \emph{FG}, because an iteration of \emph{FG} samples all $N$ entries.
After an iteration of \emph{FG}, $N$ is also the maximum number of \emph{distinct} entries that \tool can predict ahead. 
Spread over $\theta(N)$ iterations, the overhead associated with an iteration of \emph{FG} armortizes to $\theta(1)$ over time.
In average, the re-computation and armortization together triple \tool's expected iteration cost over time. 
Tripling also multiplies overall time-complexity by only a constant. 
Without loss of generality, \tool posseses $\theta(1)$ iteration cost even when \tool performs iterations of \emph{FG} (up to) a constant number of times for every $\theta(N)$ iteraions.
%\tool achieves the lowest possible asymptotic $\theta(1)$ iteration cost.

%A special case happens when the number of iterations 
Our evaluation will illustrate that, despite tripling the itearation cost, \tool still returns the best optimizations within the shortest time.
\end{proof}


\emph{Theorem 7.}
\tool has $\theta(N + min(M,N) + nDims*(nRows+nCols))$ space-complexity and is asymptotically as compact as any memory-less gradient method.
\begin{proof}
$U^{s}$ and $\bar{m}_U^{s}$ are $nRows$-by-$nDims$ matrices.
$V^{s}$ and $\bar{m}_V^{s}$ are $nDims$-by-$nCols$ matrices.

\tool also stores the indices of entries that \tool is going to sample in the future; these indices take $\theta(min(E[M],N))$ space.
When $M > N$, $min(M,N)$ returns $N$; and space-complexity becomes $\theta(N + N + nDims*(nRows+nCols))$.
When $N > M$, space complexity becomes $O(N + N + nDims*(nRows+nCols))$.

The extra matrices and indices that \tool stores does not asymptotically increase the most-compact possible space-complexity (\emph{Theorem 2}).
Both full deterministic gradient \emph{FG} and stochastic gradient \emph{SG} are memory-less methods and thus they also achieve the most-compact possible space-complexity in \emph{Theorem 2}.
Indeed, \tool is as compact as any memory-less method because the space-complexity does not become any more compact than what is proved in \emph{Theorem 2}.
Our evaluation will illustrate that the actual memory usage are similar among \tool, \emph{FG} and \emph{SG}.
\end{proof}
